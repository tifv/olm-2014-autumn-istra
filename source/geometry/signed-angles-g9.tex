% $date: 2014-11-11
% $timetable:
%   g9r2:
%     2014-11-11:
%       1:
%   g9r1:
%     2014-11-11:
%       3:

\section*{Ориентированные углы}

% $authors:
% - Фёдор Ивлев

\definition
Ориентированным углом между прямыми $l$ и~$m$ называется такой угол, на~который
нужно против часовой стрелки повернуть прямую~$l$, чтобы она стала параллельна
прямой~$m$.
Обозначается ориентированный угол через $\angle (l, m)$.
Углы, отличающиеся на~кратное 180 число градусов, считаются равными.

\setcounter{jeolmsubproblem}{0}
\claim{Упражнения}
\par
\sp $\angle (l, m) = - \angle (m, l)$.
\qquad
\sp $\angle (l, m) + \angle (m, k) = \angle (l, k)$.
\par
\sp \(
    \angle (AC, CB) = \angle (AD, DB)
\quad\Leftrightarrow\quad
    \)точки $A$, $B$, $C$ и~$D$ лежат на~одной окружности.
% Заметим, что если считать, что прямая AA это касательная в точке A
% к окружности, то этот факт верен всегда и никак не зависит от расположения
% точек на окружности.
\par
\sp \(
    \angle (l, m) = \angle (l, k)
\quad\Leftrightarrow\quad
    m \parallel k
\).

\begin{problems}

\item
\sp
Две окружности пересекаются в~точках $P$ и~$Q$.
Через $P$ проходит прямая~$AB$, причем $A$ лежит на~первой окружности,
а~$B$~--- на~второй.
Через $Q$ проходит прямая~$CD$, причем $C$ лежит на~первой окружности,
а~$D$~--- на~второй.
Докажите, что $AC \parallel BD$.
\\
\sp
Дан вписанный шестиугольник $ABCDEF$.
Пусть прямые $AB$ и~$DE$ пересекаются в~точке~$Q$, прямые $BC$ и~$EF$~---
в~точке~$P$, а~$CD$ с~$FA$~--- в~точке~$R$.
Проведем описанную окружность треугольника $PCF$ и~продлим прямые $CD$ и~$FA$
до~вторичного пересечения с~этой окружностью в~точках $U$ и~$V$ соответственно.
Докажите, что соответственные стороны треугольников $AQD$ и~$VPU$ параллельны.
\\
\sp\emph{(теорема Паскаля)}
Выведите из~этого, что точки $P$, $Q$ и~$R$ лежат на~одной прямой.

\item
Даны окружности $S_1$, $S_2$ и~$S_3$, проходящие через точку~$X$.
Вторая точка пересечения окружностей $S_1$ и~$S_2$~--- точка~$P$,
$S_2$ и~$S_3$~--- точка~$Q$, $S_3$ и~$S_1$~--- точка~$R$.
На~окружности~$S_1$ выбрана произвольная точка~$A$.
Вторая точка пересечения прямой~$AP$ с~$S_2$~--- точка~$B$,
прямой~$AR$ с~$S_3$~--- точка~$C$.
Докажите, что $B$, $C$ и~$Q$ лежат на~одной прямой.

\item
На~окружности даны точки $A$, $B$, $C$, $D$.
$M$~---середина дуги~$AB$.
Обозначим точки пересечения хорд~$MC$ и~$MD$ с~хордой~$AB$ через~$E$ и~$K$.
Докажите, что точки $K$, $E$, $C$ и~$D$ лежат на~одной окружности.

\item
Дан параллелограмм $ABCD$.
Прямая пересекает его стороны $AB$, $BC$, $CD$ и~$AD$ в~различных точках
$E$, $F$, $G$, $H$ соответственно.
Описанные окружности треугольников $AEF$ и~$AGH$ пересекаются вторично
в~точке~$P$, а~описанные окружности треугольников $CEF$ и~$CGH$ пересекаются
вторично в~точке~$Q$.
Докажите, что прямая~$PQ$ делит отрезок~$BD$ пополам.
% Saudi Aravi BMO TST 2014.2.3

%\item
%Дан правильный треугольник $ABC$.
%Через вершину~$B$ проводится произвольная прямая~$l$.
%Из~точек $A$ и~$C$ на~прямую~$l$ опускаются перпендикуляры $AA'$ и~$CC'$.
%Если точки $A'$ и~$C'$ различны, то~строятся правильные треугольники $A'C'P$
%и~$A'C'Q$, смотрящие относительно $l$ в~разные стороны.
%Найдите ГМТ $P$ и~$Q$.

\item
Даны 4~прямые общего положения.
Всеми возможными способами выкидывается одна из~них, и~берется описанная
окружность оставшегося треугольника.
Докажите, что четыре таких окружности проходят через одну точку.
Эта точка называется \emph{точкой Микеля} для этой четверки прямых
(или для четырехугольника, образованного этими прямыми).

\item
В~треугольнике $ABC$ вписанная окружность с~центром~$I$ касается его сторон
$AB$ и~$BC$ в~точках $C_1$ и~$A_1$ соответственно.
Прямая~$AI$ пересекает прямую~$A_1 C_1$ в~точке~$K$.
\\
\sp Докажите, что угол $CKA$ прямой.
\\
\sp
Докажите, что точка~$K$ лежит на~средней линии треугольника $ABC$,
параллельной стороне~$AB$.
\\
\sp
Окружность с~центром~$O$ вписана в~четырехугольник $ABCD$ и~касается его
непараллельных друг другу сторон $BC$ и~$AD$ в~точках $E$ и~$F$ соответственно.
Пусть прямая~$AO$ и~отрезок~$EF$ пересекаются в~точке~$K$,
прямая~$DO$ и~отрезок~$EF$~--- в~точке~$N$,
а~прямые $BK$ и~$CN$~--- в~точке~$M$.
Докажите, что точки $O$, $K$, $M$ и~$N$ лежат на~одной окружности.
%ВМО 94.4.10.3 (М. Сонкин)

\item
Окружности $\omega$ и~$\gamma$ касаются друг друга внутренним образом
в~точке~$A$, причем $\gamma$ находится внутри~$\omega$.
Хорда~$BC$ окружности~$\omega$ касается $\gamma$ в~точке~$P$.
Прямые $AB$ и~$AC$ вторично пересекают~$\gamma$ в~точках $B'$ и~$C'$
соответственно.
\\
\sp
Докажите, что прямые $B'C'$ и~$BC$ параллельны.
(подсказка: воспользуйтесь тем, что у~обеих окружностей в~точке~$A$ общая
касательная)
\\
\sp\emph{Лемма Архимеда.}
Докажите, что прямая~$AP$ проходит через середину дуги~$BC$
окружности~$\omega$, не~содержащей точку~$A$.

\item
Точки $O_1$ и~$O_2$~--- центры соответственно описанной и~вписанной окружностей
равнобедренного треугольника $ABC$ ($AB = BC$).
Окружности, описанные около треугольников $ABC$ и~$O_1 O_2 A$, пересекаются
в~точках $A$ и~$D$.
Докажите, что прямая~$BD$ касается окружности, описанной около
треугольника~$O_1 O_2 A$.
%ВМО 97.4.10.7 (М. Сонкин)

\item
В~треугольнике $ABC$ проведена биссектриса~$B B_1$.
Из~точки~$B_1$ восстанавливается перпендикуляр к~стороне~$BC$ и~продлевается
до~пересечения с~дугой~$BC$ описанной окружности в~точке~$K$.
Перпендикуляр из~$B$ на~$AK$ пересекает~$AC$ в~точке~$L$.
Докажите, что $L$, $K$ и~середина дуги~$AC$ лежат на~одной прямой.

\item
Дан треугольник $ABC$ и~внутри него точка~$P$, отличная от~точки пересечения
высот.
\\
\sp
Докажите, что окружности, проходящие через середины сторон треугольников
$ABP$, $BCP$, $CAP$, $ABC$, проходят через одну точку.
\\
\sp
Докажите, что описанная окружность педального треугольника точки~$P$
(треугольника, образованного проекциями точки на~стороны исходного
треугольника) также проходит через эту точку.

\end{problems}

