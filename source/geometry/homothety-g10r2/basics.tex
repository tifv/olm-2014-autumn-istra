% $date: 2014-11-17
% $timetable:
%   g10r2:
%     2014-11-17:
%       1:

\section*{Гомотетия}

% $authors:
% - Дарья Сербина

\begin{problems}

\item
Докажите, что точки, симметричные произвольной точке относительно середин
сторон квадрата, являются вершинами некоторого квадрата.

\item
Продолжения боковых сторон $AB$ и~$CD$ трапеции $ABCD$ пересекаются
в~точке~$K$, а~её~диагонали~--- в~точке~$L$.
Докажите, что точки $K$, $L$, $M$ и~$N$, где $M$ и~$N$~--- середины оснований
$BC$ и~$AD$, лежат на~одной прямой.

\item
На~окружности фиксированы точки $A$ и~$B$, а~точка~$C$ движется по~этой
окружности.
Найдите геометрическое место точек пересечения медиан треугольников $ABC$.

\item
Окружность~$S$ касается равных сторон $AB$ и~$BC$ равнобедренного
треугольника $ABC$ в~точках $P$ и~$K$, а~также касается внутренним образом
описанной окружности треугольника $ABC$.
Докажите, что середина отрезка~$PK$ является центром вписанной окружности
треугольника $ABC$.

\item
\sp
Вписанная окружность треугольника $ABC$ касается стороны~$AC$ в~точке~$D$,
$DM$~--- её~диаметр.
Прямая~$BM$ пересекает сторону~$AC$ в~точке~$K$.
Докажите, что $AK = DC$.
\\
\sp
В~окружности проведены перпендикулярные диаметры $AB$ и~$CD$.
Из~точки~$M$, лежащей вне окружности, проведены касательные к~окружности,
пересекающие прямую~$AB$ в~точках $E$ и~$H$, а~также прямые $MC$ и~$MD$,
пересекающие прямую~$AB$ в~точках~$F$ и~$K$.
Докажите, что $EF = KH$.

\item
Внутри угла~$A$ выбрана произвольная точка~$M$.
Постройте с~помощью циркуля и~линейки окружность, проходящую через~$M$
и~касающуюся сторон угла~$A$.

\item
В~параллелограмме $ABCD$ на~диагонали~$AC$ отмечена точка~$K$.
Окружность~$S_1$ проходит через точку~$K$ и~касается прямых $AB$ и~$AD$
($S_1$ вторично пересекает диагональ~$AC$ на~отрезке~$AK$).
Окружность~$S_2$ проходит через точку~$K$ и~касается прямых $CB$ и~$CD$
($S_2$ вторично пересекает диагональ~$AC$ на~отрезке~$KC$).
Докажите, что при всех положениях точки~$K$ на~диагонали~$AC$ прямые,
соединяющие центры окружностей $S_1$ и~$S_2$, будут параллельны между собой.

\end{problems}

