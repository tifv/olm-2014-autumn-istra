% $date: 2014-11-20
% $timetable:
%   g10r2:
%     2014-11-20:
%       1:

\section*{Гомотетическая добавка}

% $authors:
% - Дарья Сербина

\begin{problems}

\item
В~треугольнике $ABC$ точка~$I$~--- центр вписанной окружности,
$I'$~--- центр вневписанной в~угол~$C$ окружности;
$L$ и~$L'$~--- точки, в~которых сторона~$AB$ касается этих окружностей.
Докажите, что прямые $IL'$, $LI'$ и~высота~$CH$ треугольника $ABC$ пересекаются
в~одной точке.

\item
Ортоцентр треугольника $ABC$ отразили симметрично относительно сторон его
ортотреугольника, получили точки $A_1$, $B_1$, $C_1$.
Докажите, что прямые $A A_1$, $B B_1$, $C C_1$ пересекаются в одной точке.

\item
Две окружности пересекаются в точках $A$ и $B$.
Прямая, проходящая через $B$, пересекает первую окружность в точке~$C$,
а вторую в точке~$D$; $M$ --- середина~$CD$.
Найдите ГМТ $M$.

\item
Докажите, что любой выпуклый многоугольник $\Phi$ содержит два непересекающихся
многоугольника $\Phi_1$и $\Phi_2$, подобных $\Phi$ с коэффициентом $1/2$.

\end{problems}

