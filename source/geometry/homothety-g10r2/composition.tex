% $date: 2014-11-18
% $timetable:
%   g10r2:
%     2014-11-18:
%       1:

\section*{Композиция гомотетий}

% $authors:
% - Дарья Сербина

\begin{problems}

\item\emph{Лемма о~трех колпаках.}
Общие внешние касательные к~парам окружностей $S_1$ и~$S_2$, $S_2$ и~$S_3$,
$S_3$ и~$S_1$ пересекаются в~точках $A$, $B$ и~$C$ соответственно.
Докажите, что точки $A$, $B$ и~$C$ лежат на~одной прямой.

\item
Из~вершины~$A$ треугольника $ABC$ проведён луч~$AM$, лежащий внутри
треугольника ($M$ лежит на~$BC$).
Обозначим через $\gamma_1$, $\gamma_2$ вписанную и~вневписанную окружности
треугольника $AMB$ соответственно (берется окружность, касающаяся
стороны~$MB$).
Аналогично для треугольника $ACM$ определены окружности $\omega_1$, $\omega_2$.
Докажите, что общая внешняя касательная к~окружностям $\gamma_1$ и~$\omega_1$,
отличная от~$BC$ и~общая внешняя касательная к~окружностям $\gamma_2$ и~$\omega_2$, отличная от~$BC$, пересекаются на~прямой $BC$.

\item
Пусть $M_A$, $M_B$, $M_C$~--- середины сторон треугольника $ABC$.
Точки $A'$, $B'$, $C'$~--- суть точки касания:
\\
\sp
вписанных окружностей треугольников $M_A M_B C$, $A M_B M_C$ и~$M_A B M_C$
со~сторонами треугольника $M_A M_B M_C$;
\\
\sp
вписанной окружности треугольника $M_A M_B M_C$ с~его сторонами.
\\
Докажите, что прямые $AA'$, $BB'$ и~$CC'$ пересекаются в~одной точке.

\item
Внутри треугольника расположены окружности
$\alpha$, $\beta$, $\gamma$, $\delta$ одинакового радиуса, причем каждая
из~окружностей $\alpha$, $\beta$, $\gamma$ касается двух сторон треугольника
и~окружности~$\delta $.
Докажите, что центр окружности~$\delta$ принадлежит прямой, проходящей через
центры вписанной и~описанной окружностей данного треугольника.

\item
Впишите в~треугольник две равные окружности, каждая из~которых касается двух
сторон треугольника и~другой окружности.

\item
В~четырехугольнике $ABCD$ отметили точку~$P$.
В~каждый из~треугольников $ABP$, $BCP$, $CDP$, $DAP$ вписали по~окружности
с~центрами $O_1$, $O_2$, $O_3$, $O_4$ соответственно.
Оказалось, что каждая из~окружностей касается двух соседних.
Докажите, что прямые $O_1 O_2$, $O_3 O_4$ и~$AC$ пересекаются в~одной точке
либо параллельны.

\end{problems}

