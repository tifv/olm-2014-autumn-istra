% $date: 2014-11-19
% $timetable:
%   g10r2:
%     2014-11-19:
%       1:

\section*{Поворотная гомотетия}

% $authors:
% - Дарья Сербина

\begingroup
    \def\const{\operatorname{const}}

\begin{problems}

\item
\sp
Пусть $P$~--- точка пересечения прямых $AB$ и~$A_1 B_1$.
Докажите, что если среди точек $A$, $B$, $A_1$, $B_1$ и~$P$ нет совпадающих,
то~общая точка описанных окружностей треугольников $P A A_1$ и~$P B B_1$
является центром поворотной гомотетии, переводящей точку~$A$ в~$A_1$,
а~точку~$B$ в~$B_1$, причем такая поворотная гомотетия единственна.
\\
\sp
Докажите, что центром поворотной гомотетии, переводящей отрезок~$AB$
в~отрезок~$BC$, является точка пересечения окружности, проходящей через
точку~$A$ и~касающейся прямой~$BC$ в~точке~$B$,
и~окружности, проходящей через точку~$C$ и~касающейся прямой~$AB$ в~точке~$B$.

\item
Окружности $S_1$ и~$S_2$ пересекаются в~точках $A$ и~$B$.
При поворотной гомотетии с~центром~$A$, переводящей $S_1$ в~$S_2$, точка~$M_1$
окружности~$S_1$ переходит в~$M_2$.
Докажите, что прямая $M_1 M_2$ проходит через точку $B$.

\item
\sp
По~лучам, имеющим общее начало, с~постоянными неравными скоростями двигаются
точки $A$ и~$B$.
Докажите, что есть две точки плоскости, из~под которых отрезок $AB$ виден под
постоянным углом.
\\
\sp
По~двум неравным окружностям с~равными угловыми скоростями двигаются точки $A$
и~$B$.
Докажите, что существует точка $P$ плоскости такая, что $\angle APB = \const$.

\item
Две окружности пересекаются в~точках $A$ и~$B$, а~хорды $AM$ и~$AN$ касаются
этих окружностей.
Треугольник $MAN$ достроен до~параллелограмма $MANC$ и~отрезки $BN$ и~$MC$
разделены точками $P$ и~$Q$ в~равных отношениях.
Докажите, что $\angle APQ = \angle ANC$.

\item
Дана полуокружность с~диаметром~$AB$.
Для каждой точки~$X$ этой полуокружности на~луче~$XA$ откладывается точка~$Y$
так, что $XY = k \cdot XB$.
Найдите ГМТ $Y$.

\item
Точки $A_2$, $B_2$ и~$C_2$~--- середины высот $A A_1$, $B B_1$ и~$C C_1$
остроугольного треугольника $ABC$.
Найдите сумму углов
$\angle B_2 A_1 C_2$, $\angle C_2 B_1 A_2$ и~$\angle A_2 C_1 B_2$.

\item
Точка~$M$ выбрана на~стороне~$BC$ так, что $MA = MC$.
Биссектриса угла~$AMB$ пересекает описанную окружность треугольника $ABC$
в~точке~$K$.
Докажите, что прямая, проходящая через центры вписанных окружностей
треугольников $AKM$ и~$BKM$, перпендикулярна биссектрисе угла~$AKB$.

\item
Постройте четырехугольник $ABCD$ по~сумме $\angle B + \angle D$ и~сторонам
$a = AB$, $b = BC$, $c = CD$ и~$d = DA$.

\end{problems}

\endgroup % \def\const

