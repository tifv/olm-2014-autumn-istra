% $date: 2014-11-10
% $timetable:
%   g11r2:
%     2014-11-10:
%       3:

% $caption: Разнобой (геометрия)

\section*{Разнобой}

% $authors:
% - Андрей Кушнир

\begin{problems}

\item
В~треугольнике $ABC$ проведены чевианы $A B_1$, $A C_1$, симметричные
относительно биссектрисы угла~$A$.
Докажите, что центры описанных окружностей треугольников $A B B_1$, $A B C_1$,
$A C B_1$, $A C C_1$ лежат на~одной окружности.

\item
Точки $B'$, $C'$ симметричны вершинам $B$, $C$ треугольника $ABC$ относительно
сторон $AC$, $AB$ соответственно.
Описанные окружности треугольников $ABB'$ и~$ACC'$ пересекаются в~точке~$D$.
Докажите, что $AD$ проходит через центр описанной окружности треугольника.

\item
На~высотах $B B_1$, $C C_1$ остроугольного треугольника $ABC$ отмечены точки
$B_2$, $C_2$, так что $\angle A B_2 C = \angle A C_2 B = 90^{\circ}$.
Докажите, что $A B_2 = A C_2$.

\item
$ABCD$~--- выпуклый четырехугольник.
$E$, $F$, $K$, $M$~--- соответственно середины отрезков $AB$, $CD$, $AD$, $AK$.
Оказалось, что $AF$, $BK$, $DE$, $CM$ пересекаются в~одной точке~$X$.
Докажите, что площадь четырехугольника $BCFX$ равна половине площади $ABCD$.

\item
Треугольник $ABC$ лежит внутри окружности~$\omega$.
В~криволинейный треугольник, образованный продолжениями сторон $AB$, $AC$
за~точку~$A$ и~дугой $\omega$, вписана окружность, касающаяся $\omega$
в~точке~$A_1$.
Аналогично определяются $B_1$, $C_1$.
Докажите, что $A A_1$, $B B_1$, $C C_1$ пересекаются в~одной точке.

\item
На~продолжении стороны~$BC$ отмечены точки $P$ и~$Q$, такие что $AP = AQ = p$,
где $p$~--- полупериметр треугольника.
Докажите, что описанная окружность треугольника $APQ$ касается вневписанной.

\end{problems}

