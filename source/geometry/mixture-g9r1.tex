% $date: 2014-11-10
% $timetable:
%   g9r1:
%     2014-11-10:
%       2:

\section*{Разнобой по геометрии}

% $authors:
% - Фёдор Ивлев

\begin{problems}

\item
\sp
Постройте общую касательную к~двум окружностям.
\\
\sp Дана окружность и~точка~$P$.
Постройте в~окружности хорду~$AB$ заданной длины и~с~заданной длиной
перпендикуляра на~неё из~точки~$P$.
% Вестник элементарной математики

\item
Две окружности пересекаются в~точках $E$ и~$F$.
Прямая~$l$ пересекает первую окружность в~точках $A$ и~$B$, вторую~---
в~точках $C$ и~$D$ так, что точка~$E$ лежит внутри треугольника $ADF$,
а~точки $B$ и~$C$~--- на~отрезке~$AD$.
Оказалось, что $AB = CD$.
Докажите, что $BE \cdot DF = CE \cdot AF$.
% Питерская городская олимпиада, 2011.10.5, первый тур (Д. Максимов)

\item
В~окружность вписан пятиугольник $ABCDE$.
Отрезки $AC$ и~$BD$ пересекаются в~точке~$K$.
Отрезок~$CE$ касается описанной окружности треугольника $ABK$ в~точке~$N$.
Найдите $\angle CNK$, если известно, что $\angle ECD = 40^\circ$.
% Питерская городская олимпиада, 2011.11.4, первый тур (А. Смирнов)

\item
Обозначим центр вписанной окружности треугольника $ABC$ через~$I$,
а~через~$L$~--- середину дуги~$AB$ его описанной окружности.
Из~$L$ опустили перпендикуляр на~прямую~$AI$, который пересек сторону~$AC$
в~точке~$K$.
Докажите, что $KI \parallel AB$.
% районный этап какого-то недавнего года.

\item
Из~центра~$O$ описанной окружности треугольника $ABC$ опустили перпендикуляры
$OM$ и~$ON$ на~стороны $AB$ и~$BC$, а~затем построили параллелограмм $MONK$.
Докажите, что точки $B$, $K$ и~ортоцентр~$H$ треугольника $ABC$ лежат
на~одной прямой.
% Вестник элементарной математики №346, задача №344

\item
Докажите, что в~прямоугольном треугольнике высота, проведённая из~прямого угла,
равна сумме радиуса вписанной в~треугольник окружности и~радиусов вписанных
окружностей треугольников, на~которые эта высота делит исходный треугольник.
% Вестник элементарной математики №343, задача №325

\item
Два квадрата $ABCD$ и~$AKLM$ имеют общую вершину и~одинаково ориентированы.
Докажите, что $BK \perp DM$.

\end{problems}

