% $date: 2014-11-13
% $timetable:
%   g11r2:
%     2014-11-13:
%       1:
%   g11r1:
%     2014-11-13:
%       2:

\section*{Касательные к сферам}

% $authors:
% - Фёдор Нилов

\begin{problems}

\item
В~четрырехгранный угол вписана сфера.
Докажите, что суммы его противоположных плоских углов равны.

\item
Дана плоскость и~две точки $A$ и~$B$, лежащие по~одну сторону от~нее.
Рассматриваются всевозможные сферы,
проходящие через $A$ и~$B$, и~касающиеся данной плоскости.
Найдите ГМТ точек касания.

\item
Докажите, что на~ребрах тетраэдра можно написать по~положительному числу так,
чтобы сумма чисел, записанных на~ребрах каждой грани, равнялась~бы площади этой
грани.

\item
Стороны треугольника равны $a$, $b$, $c$.
Три шара попарно касаются друг друга внешним образом и~плоскости треугольника
в~его вершинах.
Найдите радиусы шаров.

\item
Около сферы описан пространственный четырехугольник.
Докажите, что четыре точки касания лежат в~одной плоскости.

\item
Дан тетраэдр $ABCD$.
Сфера, проходящая через точки $A$, $B$ и~$C$ пересекает ребра
$DA$, $DB$ и~$DC$ в~точках $A_1$, $B_1$ и~$C_1$ соответственно.
Точки $A_1$, $B_1$ и~$C_1$ отразили относительно середин соответствующих ребер,
на которых они лежат, и~получили точки $A_2$, $B_2$ и~$C_2$.
Докажите, что центр сферы, описанной около тетраэдра $D A_2 B_2 C_2$,
равноудален от~точек $A$, $B$ и~$C$.

\item
Точка~$O$~--- основание высоты четырехугольной пирамиды.
Сфера с~центром~$O$ касается всех боковых граней пирамиды.
Точки $A$, $B$, $C$ и~$D$ взяты последовательно по~одной на~боковых ребрах
пирамиды так, что отрезки $AB$, $BC$ и~$CD$ проходят через три точки касания
сферы с~гранями.
Докажите, что отрезок~$AD$ проходит через четвертую точку касания.

\item
Сферы $S_1$, $S_2$ и~$S_3$ касаются друг друга внешним образом и~некоторой
плоскости в~точках $A$, $B$ и~$C$.
Сфера~$S$ касается сфер $S_1$, $S_2$ и~$S_3$ внешним образом и~данной плоскости
в~точке~$D$.
Докажите, что проекции точки~$D$ на~стороны треугольника $ABC$ являются
вершинами правильного треугольника.

\item
Дан тетраэдр $ABCD$.
Вписанная в~него сфера~$S$ касается грани $ABC$ в~точке~$T$.
Сфера~$S'$ касается грани $ABC$ в~точке~$T'$, а~также продолжений граней
$ABD$, $BCD$ и~$CAD$.
Докажите, что прямые $AT$ и~$AT'$ симметричны относительно биссектрисы
угла $BAC$.

\item
Сфера вписана в~четырехугольную пирамиду $ABCDS$.
Пусть $T$~--- точка касания этой сферы с~основанием $ABCD$.
Докажите, что проекции точки $T$ на~стороны четырехугольника $ABCD$ лежат
на~одной окружности.

\item
Сфера с~центром в~плоскости основания $ABC$ тетраэдра $SABC$ проходит через
вершины $A$, $B$ и~$C$ и~вторично
пересекает ребра $SA$, $SB$ и~$SC$ в~точках $A_1$, $B_1$ и~$C_1$,
соответственно.
Плоскости, касающиеся сферы в~точках
$A_1$, $B_1$ и~$C_1$, пересекаются в~точке $O$.
Докажите, что $O$~--- центр сферы, описанной около тетраэдра $SA_1B_1C_1$.

%\item
%На~ребрах произвольного тетраэдра выбрано по~точке.
%Через каждую тройку точек, лежащих на~ребрах с~общей вершиной, проведена
%плоскость.
%Докажите, что если три из~четырех проведенных плоскостей касаются вписанного
%в~тетраэдр шара, то~и~четвертая плоскость также его касается.

\end{problems}

