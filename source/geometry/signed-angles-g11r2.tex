% $date: 2014-11-11
% $timetable:
%   g11r2:
%     2014-11-11:
%       3:

\section*{Ориентированные углы}

% $authors:
% - Андрей Кушнир

\begin{problems}

\item
$ABC$~--- треугольник.
Докажите, что если точки $A_1$, $B_1$, $C_1$ лежат на~прямых $BC$, $CA$, $AB$,
то~окружности $A_1 C B_1$, $B_1 A C_1$, $C_1 B A_1$ имеют общую точку.

\item
Четыре окружности пересекают друг друга по~циклу (т.~е. первая вторую, вторая
третью, третья четвертую, четвертая первую) в~четырех парах точек.
Известно, что из~этих четырех пар можно выбрать по~одной точке, так чтобы они
лежали на~одной окружности или прямой.
Докажите, что оставшиеся четыре точки тоже лежат на~одной окружности или
прямой.

\item
Из~основания высоты треугольника опустили перпендикуляры на~его стороны
и~высоты.
Докажите, что основания опущенных перпендикуляров лежат на~одной прямой.

\item\emph{(Обобщённая теорема Симсона)}
Из~точки~$P$ на~описанной окружности треугольника $ABC$ провели
прямые $l_a$, $l_b$, $l_c$ такие, что
$\angle (l_a, BC) \equiv \angle (l_b, CA) \equiv \angle (l_c, AB)$.
Докажите, что точки пересечения $l_a$ и~$BC$, $l_b$ и~$CA$, $l_c$ и~$AB$ лежат
на~одной прямой.
(Если равные углы из~условия прямые, то~эта прямая называется прямой Симсона)

\item
Точку на~описанной окружности треугольника $ABC$ отразили симметрично
относительно его сторон.
Докажите, что три полученные таким образом точки лежат на~одной прямой,
проходящей через ортоцентр треугольника $ABC$.

\item\emph{(Точка Микеля)}
Даны четыре прямые общего положения.
Вокруг треугольников, образованных всевозможными тройками прямых, описаны
окружности.
Докажите, что они имеют общую точку.

\item
Даны два неколлинеарных направленных отрезка.
Докажите, что существует поворотная гомотетия (т.~е. композиция гомотетии
и~поворота с~одним и~тем~же центром), переводящая один отрезок в~другой.
Как построить ее центр?

\item
Стороны выпуклого пятиугольника продолжены до~пересечения так, что образовалась
пятиконечная звезда (пентаграмма).
Вокруг каждого из~лучей этой звезды, т.~е. треугольника, примыкающего к~одной
из~сторон пятиугольника, описана окружность.
Докажите, что точки пересечения соседних окружностей, отличные от~вершин
пятиугольника, лежат на~одной окружности.

\end{problems}

