% $date: 2014-11-13
% $timetable:
%   g9r1:
%     2014-11-13:
%       3:

\section*{Маленькая добавка}

% $authors:
% - Фёдор Ивлев

\begin{problems}

\item
Даны 5 прямых общего положения.
Докажите, что пять точек Микеля всевозможных четверок из~этих прямых лежат
на~одной окружности.

\item
Пусть $A_1$, $B_1$, $C_1$~--- середины дуг $BAC$, $ABC$, $ACB$ описанной
окружности треугольника $ABC$.
Докажите, что ортоцентры треугольников $A_1 B C_1$, $B_1 C A_1$, $C_1 A B_1$
образуют треугольник, подобный $A_1 B_1 C_1$.

\item
На~диагоналях выпуклого четырехугольника $ABCD$ построены правильные
треугольники $ACB'$ и~$BDC'$, причем точки $B$ и~$B'$ лежат по~одну сторону
от~$AC$, а~точки $C$ и~$C'$ лежат по~одну сторону от~$BD$.
Найдите $\angle BAD + \angle CDA$, если известно, что $B'C'= AB + CD$.

\item
Пусть $ABC$~--- правильный треугольник.
На~его стороне~$AC$ выбрана точка~$T$, а~на~дугах $AB$ и~$BC$ его описанной
окружности выбраны точки $M$ и~$N$ соответственно так, что
$MT$ параллельно~$BC$ и~$NT$ параллельно~$AB$.
Отрезки $AN$ и~$MT$ пересекаются в~точке~$X$, а~отрезки $CM$ и~$NT$~---
в~точке $Y$.
Докажите, что периметры многоугольников $AXYC$ и~$XMBNY$ равны.
%ВМО 2011.5.9.7

\item
Пусть $A A_1$, $B B_1$, $C C_1$~--- высоты остроугольного треугольника $ABC$;
$O_A$, $O_B$, $O_C$~--- центры вписанных окружностей треугольников
$A B_1 C_1$, $B C_1 A_1$, $C A_1 B_1$ соответственно;
$T_A$, $T_B$, $T_C$~--- точки касания вписанной окружности $ABC$ со~сторонами
$BC$, $CA$, $AB$ соответственно.
Докажите, что все стороны шестиугольника $T_A O_B T_C O_A T_B O_C$ равны.

\end{problems}

