% $date: 2014-11-18
% $timetable:
%   g9r1:
%     2014-11-18:
%       3:

\section*{Гомотетия}

% $authors:
% - Фёдор Ивлев

% Разобрать:
% - то, что вектор переходит в себя, помноженного на число.
% - задачу про трапецию.
% - лемму Архимеда.
% - задачу про перспективные треугольники с попарно параллельными сторонами.

\begingroup
    \def\ov{\overline}

\definition
\emph{Гомотетией} с~центром в~точке~$O$ и~коэффициентом $k \neq 0$ называется
преобразование плоскости, которое переводит каждую точку~$P$ в~такую
точку~$P'$, что $\ov{OP'} = k \cdot \ov{OP}$.

\begin{problems}

\item
Точки $A$, $B$ и~$C$ лежат на~окружности~$\Gamma$.
Найдите ГМТ пересечения медиан треугольников $ABC$, если точки $A$ и~$B$
фиксированы, а~$C$ двигается по~$\Gamma$.

\item
В~треугольнике $ABC$ проведены медианы $A A_1$, $B B_1$, $C C_1$ и~взята
произвольная точка~$P$.
Через точку~$A$ провели прямую $l_a$, параллельную прямой~$P A_1$.
Аналогично определяются прямые $l_b$ и~$l_c$.
Докажите, что прямые $l_a$, $l_b$, $l_c$ пересекаются в~одной точке.
% Прасолов 19.4

\item
В~треугольнике $ABC$ медианы $A A_0$, $B B_0$, $C C_0$ пересекаются
в~точке~$M$, высоты $A A_1$, $B B_1$, $C C_1$ пересекаются в~точке~$H$,
а~$O$~--- вообще центр описанной окружности.
\\
\sp
Докажите, что точки, симметричные~$H$ относительно точки~$A_0$ и~относительно
стороны~$BC$ лежат на~описанной окружности треугольника.
\\
\sp
Докажите, что точки $A_0$, $B_0$, $C_0$, $A_1$, $B_1$, $C_1$ лежат на~одной
окружности (\emph{окружность Эйлера}), радиус которой вдвое меньше радиуса
описанной окружности треугольника $ABH$.
Какие ещё три точки лежат на~этой окружности?
\\
\sp
Докажите, что радиус вписанной окружности хотя~бы вдвое меньше радиуса
описанной окружности.
\\
\sp
Докажите, что точки $H$, $O$, $M$ и~центр окружности Эйлера лежат на~одной
прямой и~найдите отношения, в~которых делят две последние точки отрезок
с~концами в~первых двух.

\item
\emph{Построим сначала что-то~похожее, а~потом подгоним.}
\\
\sp
Даны угол $ABC$ и~точка~$M$ внутри его.
Постройте окружность, касающуюся сторон угла и~проходящую через точку~$M$.
% Прасолов 19.16
\\
\sp
Впишите в~треугольник две равные окружности, каждая из~которых касается двух
сторон треугольника и~другой окружности.
% Прасолов 19.17

\item
Обозначим точки касания вписанной и~вневписанной окружностей со~стороной~$AC$
треугольника $ABC$ через $P$ и~$Q$ соответственно.
Докажите, что прямая~$BQ$ проходит через точку, диаметрально противоположную
точке~$P$ на~вписанной окружности.

\item
Внутри треугольника расположены окружности
$\alpha$, $\beta$, $\gamma$, $\delta$ одинакового радиуса, причем каждая
из~окружностей $\alpha$, $\beta$, $\gamma$ касается двух сторон треугольника
и~окружности $\delta$.
Докажите, что центр окружности $\delta$ принадлежит прямой, проходящей через
центры вписанной и~описанной окружностей данного треугольника.

\item
Пусть $M_A$, $M_B$, $M_C$~--- середины сторон треугольника $ABC$.
Точки $A'$, $B'$, $C'$~--- суть точки касания
\\
\sp
вписанных окружностей треугольников $M_A M_B C$, $A M_B M_C$ и~$M_A B M_C$
со~сторонами треугольника $M_A M_B M_C$;
\\
\sp
вписанной окружности треугольника $M_A M_B M_C$ с~его сторонами.
\\
Докажите, что прямые $AA'$, $BB'$ и~$CC'$ пересекаются в~одной точке.

\item
Окружность~$S$ касается равных сторон $AB$ и~$BC$ равнобедренного
треугольника $ABC$ в~точках $P$ и~$Q$, а~также касается внутренним образом
описанной окружности треугольника $ABC$.
Докажите, что середина отрезка~$PQ$ является центром вписанной окружности
треугольника $ABC$.

%\item
%В~параллелограмме $ABCD$ на~диагонали~$AC$ отмечена точка~$K$.
%Окружность~$S_1$ проходит через точку~$K$ и~касается прямых $AB$ и~$AD$
%($S_1$ вторично пересекает диагональ~$AC$ на~отрезке~$AK$).
%Окружность~$S_2$ проходит через точку~$K$ и~касается прямых $CB$ и~$CD$
%($S_2$ вторично пересекает диагональ~$AC$ на~отрезке~$KC$).
%Докажите, что при всех положениях точки~$K$ на~диагонали~$AC$ прямые,
%соединяющие центры окружностей $S_1$ и~$S_2$, будут параллельны между собой.
%% (Т. Емельянова) всероссийская олимпиада окружной этап 2000-2001 год 10.2

\item
Докажите, что любой выпуклый многоугольник~$F$ содержит два непересекающихся
во~внутренних точках многоугольника $F_1$ и~$F_2$, подобных~$F$
с~коэффициентом~$1 / 2$.

%\item
%Пусть $AD$~--- биссектриса треугольника $ABC$, и~прямая~$\ell$ касается
%окружностей, описанных около треугольников $ADB$ и~$ADC$ в~точках $M$ и~$N$
%соответственно.
%Докажите, что окружность, проходящая через середины отрезков $BD$, $DC$ и~$MN$,
%касается прямой~$\ell$.
%% (Н. Седракян) Всероссийская олимпиада, окружной этап, 2000-2001 год, 11.3

\end{problems}

\endgroup % \dev\ov

