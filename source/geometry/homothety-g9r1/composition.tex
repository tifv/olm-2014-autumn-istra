% $date: 2014-11-19
% $timetable:
%   g9r1:
%     2014-11-19:
%       1:

\section*{Композиция гомотетий}

% $authors:
% - Фёдор Ивлев

\begin{problems}

\item
Высоты $A A_1$, $B B_1$, $C C_1$ треугольника $ABC$ пересекаются в~точке~$H$.
Пусть $H_A$, $H_B$, $H_C$~--- точки, симметричные $H$ относительно прямых
$B_1 C_1$, $A_1 C_1$, $A_1 B_1$ соответственно.
Докажите, что треугольники $ABC$ и~$H_A H_B H_C$ перспективны.

\item
В~параллелограмме $ABCD$ на~диагонали~$AC$ отмечена точка~$K$.
Окружность~$S_1$ проходит через точку~$K$ и~касается прямых $AB$ и~$AD$
($S_1$ вторично пересекает диагональ~$AC$ на~отрезке~$AK$).
Окружность~$S_2$ проходит через точку~$K$ и~касается прямых $CB$ и~$CD$
($S_2$ вторично пересекает диагональ~$AC$ на~отрезке~$KC$).
Докажите, что при всех положениях точки~$K$ на~диагонали~$AC$ прямые,
соединяющие центры окружностей $S_1$ и~$S_2$, будут параллельны между собой.
% (Т. Емельянова) всероссийская олимпиада окружной этап 2000-2001 год 10.2

\item
В~треугольнике $ABC$ точка~$I$~--- центр вписанной окружности,
$I'$~--- центр вневписанной в~угол~$C$ окружности;
$L$ и~$L'$~--- точки, в~которых сторона~$AB$ касается этих окружностей.
Докажите, что прямые $IL'$, $LI'$ и~высота~$CH$ треугольника $ABC$ пересекаются
в~одной точке.

\item
Даны две окружности $\omega_1$ и~$\omega_2$.
Окружность $\gamma$ касается их~внешним образом в~точках $A$ и~$B$.
Докажите, что прямая $AB$ проходит через фиксированную точку, не~зависящую
от~выбора $\gamma$.

\item
Из~вершины~$A$ треугольника $ABC$ проведён луч~$AM$, лежащий внутри
треугольника ($M$ лежит на~$BC$).
Обозначим через $\gamma_1$, $\gamma_2$ вписанная и~вневписанная окружности
треугольника $AMB$ соответственно (берется окружность, касающаяся
стороны $MB$).
Аналогично для треугольника $ACM$ определены окружности $\omega_1$, $\omega_2$.
Докажите, что общая внешняя касательная к~окружностям $\gamma_1$ и~$\omega_1$,
отличная от~$BC$, и~общая внешняя касательная к~окружностям $\gamma_2$
и~$\omega_2$, отличная от~$BC$, пересекаются на~прямой~$BC$.

\itemx{*}
Внутри выпуклого четырехугольника $ABCD$ выбрана точка~$O$.
Обозначим вписанную окружность треугольника $OAB$ через $\omega_{AB}$.
Аналогично определим окружности $\omega_{BC}$, $\omega_{CD}$ и~$\omega_{DA}$.
Оказалось, что
$\omega_{AB}$ касается $\omega_{BC}$,
$\omega_{BC}$ касается $\omega_{CD}$,
$\omega_{CD}$ касается $\omega_{DA}$
и~$\omega_{DA}$ касается $\omega_{AB}$.
Докажите, что
общая внешняя касательная к~окружностям $\omega_{AB}$ и~$\omega_{BC}$,
общая внешняя касательная к~окружностям $\omega_{CD}$ и~$\omega_{DA}$
и~прямая $AC$ пересекаются в~одной точке.

\end{problems}

