% $date: 2014-11-12
% $timetable:
%   g10r1:
%     2014-11-12:
%       1:

\section*{Ориентированные углы}

% $authors:
% - Алексей Доледенок

\definition
Ориентированным углом $\angle (l_1, l_2)$ между прямыми $l_1$ и $l_2$
называется такой угол, на~который нужно против часовой стрелки повернуть
прямую~$l_2$, чтобы она стала параллельна $l_1$.
Углы, отличающиеся на $n \cdot 180^{\circ}$, считаются равными.

\claim{Упражнения}\resetsubproblem
\begingroup \abovedisplayskip=0.5ex
\begin{flalign*} &
\subproblem
    \angle (l_1, l_2) + \angle (l_2, l_1) = 0
\, . \qquad
\subproblem
    \angle (l_1, l_2) + \angle (l_2, l_3) = \angle (l_1, l_3)
\, .  & \\ &
\subproblem
    \angle (l_1, l_2) = \angle (l_1, l_3)
\quad\Longleftrightarrow\quad
    l_2 \parallel l_3
\, . & \\ &
\subproblem
    \angle (AB, BC) = \angle (AD, DC)
\quad\Longleftrightarrow\quad
    \begin{minipage}{0.45\linewidth} \raggedright
        точки $A$, $B$, $C$, $D$ лежат на~одной окружности или прямой.
    \end{minipage}
& \end{flalign*}
\endgroup % \abovedisplayskip

\begin{problems}

\item
Две окружности пересекаются в~точках $X$ и~$Y$.
Через точки $X$ и $Y$ проведены прямые $AB$ и $CD$ соответственно, пересекающие
первую окружность в~точках $A$ и~$C$, а~вторую в~точках $B$ и $D$.
Докажите, что $AC \parallel BD$.

\item
Точки $A_1$, $B_1$, $C_1$ лежат на~прямых $BC$, $AC$, $AB$ соответственно.
Докажите, что окружности, описанные вокруг треугольников
$A B_1 C_1$, $A_1 B C_1$, $A_1 B_1 C$, пересекаются в~одной точке.

\item
В~треугольнике $ABC$ провели высоты $A A_1$, $B B_1$, $C C_1$.
Докажите, что основания перпендикуляров из~точки~$C_1$ на прямые
$A A_1$, $B B_1$, $AC$, $BC$ лежат на одной прямой.

\item\emph{Обобщенная прямая Симсона.}
Из точки~$P$ на описанной окружности треугольника $ABC$ провели прямые
$l_A$, $l_B$, $l_C$ такие, что
$\angle (l_A, BC) = \angle (l_B, CA) = \angle (l_C, AB)$.
Докажите, что точки пересечения прямых $l_A$ и $BC$, $l_B$ и $AC$, $l_C$ и $AB$
лежат на одной прямой.

\item
\subproblem\emph{Точка Микеля.}
Даны четыре прямые общего положения.
Докажите, что описанные окружности четырех образовавшихся треугольников
пересекаются в~одной точке.
\\
\subproblem
Докажите, что точка Микеля и центры окружностей лежат на одной окружности.

\item
Докажите, что в~четырехугольнике $ABCD$ окружности Эйлера треугольников
$ABC$, $ABD$, $ACD$, $BCD$ пересекаются в одной точке.

\item
Стороны выпуклого пятиугольника продлены до пересечения так, что образовалась
пятиконечная звезда.
Вокруг каждого из треугольников этой звезды, примыкающего к одной из сторон
исходного пятиугольника, описана окружность.
Докажите, что точки пересечения соседних окружностей, отличные от вершин
пятиугольника, лежат на одной окружности.

\end{problems}

