% $date: 2014-11-17
% $timetable:
%   g10r1:
%     2014-11-17:
%       2:

\section*{Геометрические решения --- зло!}

% $authors:
% - Алексей Доледенок

\begin{problems}

\item
\sp
В~треугольнике $ABC$ прямая~$AS$ симметрична медиане~$AM$ относительно
биссектрисы угла~$A$, $S \in BC$.
Докажите, что $BS : SC = AB^2 : AC^2$.
\\
\sp
Касательные в~точках $B$ и~$C$ к~описанной окружности треугольника $ABC$
пересекаются в~точке~$P$.
Докажите, что $AP$ симметрична медиане~$AM$ относительно биссектрисы угла~$A$.
\\
\sp
Окружность~$\omega_1$ проходит через вершины $A$ и~$B$ и~касается прямой~$AC$,
окружность~$\omega_2$ проходит через вершины $A$ и~$C$ и~касается прямой~$AB$.
Докажите, что общая хорда этих двух окружностей является симедианой
треугольника $ABC$.

\item\textbf{Тригонометрические теоремы Чевы и~Менелая.}
Даны точки $A_1$, $B_1$, $C_1$ на~прямых $BC$, $AC$, $AB$.
Тогда прямые $A A_1$, $B B_1$, $C C_1$ пересекаются в~одной точке
(либо параллельны) тогда и~только тогда, когда
\[
    \frac{\sin \angle A C C_1}{\sin \angle C_1 C B}
    \cdot
    \frac{\sin \angle B A A_1}{\sin \angle A_1 A C}
    \cdot
    \frac{\sin \angle C B B_1}{\sin \angle B_1 B A}
=
    1
\,;\]
Точки $A_1$, $B_1$, $C_1$ лежат на~одной прямой тогда и~только тогда, когда
это~же выражение равно $-1$ (все углы ориентированные).

\item
Докажите, что точки пересечения серединных перпендикуляров к~биссектрисам
с~прямыми, содержащими соответствующие стороны, лежат на~одной прямой.

\item
Вписанная окружность касается сторон $AB$, $BC$, $AC$ треугольника $ABC$
в~точках $C_1$, $A_1$, $B_1$ соответственно.
Внутри треугольника $ABC$ взята точка~$X$.
Прямые $AX$, $BX$, $CX$ пересекают дуги $B_1 C_1$, $A_1 C_1$, $A_1 B_1$
в~точках $A_2$, $B_2$, $C_2$ соответственно.
Докажите, что прямые $A_1 A_2$, $B_1 B_2$, $C_1 C_2$ пересекаются в~одной
точке.

\item
$A_1$, $B_1$, $C_1$~--- основания биссектрис треугольника $ABC$,
$P$ и~$Q$~--- точки пересечения прямых $A A_1$ и~$B_1 C_1$, $C C_1$ и~$A_1 B_1$
соответственно.
Докажите, что $\angle ABP = \angle CBQ$.

\item
Точки $O$ и~$H$~--- ортоцентр и~центр описанной окружности остроугольного
треугольника $ABC$; $X$, $Y$, $Z$~--- точки на~стороне~$AC$ такие, что
$HZ \parallel OX \parallel AB$, $AX = CY$.
Докажите, что $ZY = YH$.

\item
В~треугольнике $ABC$ через точку~$X$ биссектрисы~$B B_1$ угла~$B$ проведена
прямая~$l$, параллельная~$AC$.
Она пересекает сторону~$AB$ в~точке~$Y$.
Прямые $B_1Y$ и~$CX$ пересекаются в~точке~$Z$.
Докажите, что $ZB$ касается описанной окружности треугольника $ABC$.

\item
В~треугольник $ABC$ вписана окружность с~центром в~точке~$I$;
$A_1$, $B_1$, $C_1$~--- точки касания с~соответствующими сторонами;
$M$~--- середина~$AC$.
Докажите, что прямые $A_1 C_1$, $BM$, $I B_1$ пересекаются в~одной точке.

\end{problems}

