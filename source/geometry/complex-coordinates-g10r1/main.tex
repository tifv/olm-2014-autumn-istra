% $date: 2014-11-13
% $timetable:
%   g10r1:
%     2014-11-13:
%       3:

\section*{Комплексные координаты}

% $authors:
% - Алексей Доледенок

\begingroup
    \def\ov{\overline}

\claim{Замечание}
Во~всех задачах точка~$A$ имеет координату~$a$, $B$~--- координату~$b$,
$Z_i$~--- координату~$z_i$.

\begin{problems}

\item\emph{Основные факты.}
Докажите.
\\
\sp
Точка~$B$ делит отрезок~$AC$ в~отношении $p : q$.
Тогда
\[
    b
=
    a \cdot \frac{q}{p + q} + c \cdot \frac{p}{p + q}
.\]
\sp
Квадрат расстояния между точками $Z_1$ и~$Z_2$ равен
\(
    (z_1 - z_2) (\ov{z_1} - \ov{z_2})
\).
\\[0.3ex]
\sp
Точки $Z_1$, $Z_2$, $Z_3$ лежат на~одной прямой
тогда и~только тогда, когда
%\quad$\Leftrightarrow$\quad
число $\frac{z_1 - z_2}{z_2 - z_3}$ является действительным, т.~е.
\[
    \cfrac{z_1 - z_2}{z_2 - z_3}
=
    \cfrac{\ov{z_1} - \ov{z_2}}{\ov{z_2} - \ov{z_3}}
.\]
\\
\sp
Отрезки $Z_1 Z_2$ и~$Z_3 Z_4$ перпендикулярны
тогда и~только тогда, когда
число $\frac{z_1 - z_2}{z_3 - z_4}$ является мнимым (<<чисто мнимым>>), т.~е.
\[
    \cfrac{z_1 - z_2}{z_3 - z_4}
=
    -\cfrac{\ov{z_1} - \ov{z_2}}{\ov{z_3} - \ov{z_4}}
.\]
Докажите также, что в~случае, когда $Z_1$, $Z_2$, $Z_3$, $Z_4$ лежат
на~единичной окружности, это эквивалентно $z_1 z_2 + z_3 z_4 = 0$.
\\
\sp
Докажите, что прямая, проходящая через точки $A$ и~$B$ единичной окружности,
задается уравнением $z + a b \ov{z} = a + b$.
\\
\sp
$A$, $B$~--- точки на~единичной окружности, $C$~--- произвольная точка.
Найдите координату основания высоты из~вершины $C$ треугольника $ABC$.
\\
\sp
Выразите через координаты вершин координату точки пересечения медиан
треугольника $ABC$, вписанного в~единичную окружность.
\\
\sp
Выразите через координаты вершин координату ортоцентра треугольника $ABC$,
вписанного в~единичную окружность.
\\
\sp
Даны точки $A$ и~$B$ на~единичной окружности.
Выразите координаты точки пересечения касательных в~точках $A$ и~$B$ через
их~координаты.

\item
Докажите, что в~описанном четырехугольнике центр вписанной окружности лежит
на~прямой, соединяющей середины диагоналей.

\item
На~сторонах треугольника $ABC$ во~внешнюю сторону построены квадраты $ABDE$
и~$BCFG$.
Докажите, что медиана $BM$ треугольника $ABC$ перпендикулярна $DG$.

\item
На~окружности даны шесть точек.
Они произвольным образом разбиваются на~две тройки, в~первой тройке выбирается
ортоцентр $H_i$, $i = 1, \ldots, 20$, а~во~второй
\\
\sp точка пересечения высот $M_i$
\quad
\sp точка пересечения медиан $M_i$.
\\
Докажите, что все отрезки $H_i M_i$ пересекаются в~одной точке, и~найдите
отношение, в~котором они делятся этой точкой.

\item
Дан вписанный четырехугольник $ABCD$.
Рассмотрим прямую Симсона точки~$A$ относительно треугольника $BCD$,
прямую Симсона точки~$B$ относительно треугольника $ACD$ и~т.~д.
Докажите, что эти четыре прямые пересекаются в~одной точке.

\item
Дан вписанный четырехугольник $ABCD$ и~произвольная точка~$P$.
Рассмотрим отрезки, соединяющие проекции точки~$P$ на~прямые
$AB$ и~$CD$, $AC$ и~$BD$, $AD$ и~$BC$.
Докажите, что их~середины лежат на~одной прямой.

\end{problems}

\endgroup % \dev\ov

