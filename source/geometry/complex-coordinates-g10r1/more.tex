% $date: 2014-11-14
% $timetable:
%   g10r1:
%     2014-11-14:
%       2:

\section*{Ещё комплексные координаты}

% $authors:
% - Алексей Доледенок

\definition
Двойным отношением четырех точек $Z_1$, $Z_2$, $Z_3$, $Z_4$ назовем число
\[
    \frac{z_1 - z_3}{z_2 - z_3} : \frac{z_1 - z_4}{z_2 - z_4}
.\]

\begin{problems}

\item
Докажите, что точки $Z_1$, $Z_2$, $Z_3$, $Z_4$ лежат на~одной прямой или
окружности тогда и~только тогда, когда их~двойное отношение является
действительным числом.

\item
Прямоугольник $ABCD$ вписан в~окружность $\omega$ с~центром в~$O$.
Точка~$M$ на~стороне~$AD$ такова, что $AM / MD = 2$.
Луч $CM$ пересекает $\omega$ в~точке~$P$.
Докажите, что точка пересечения медиан треугольника $OPD$ лежит на~описанной
окружности треугольника $OCD$.

\item
Основание каждой высоты треугольника спроецировали на~две прилежащие стороны
треугольника.
Докажите, что шесть полученных точек лежат на~одной окружности.

\end{problems}

