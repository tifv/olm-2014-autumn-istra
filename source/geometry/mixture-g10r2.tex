% $date: 2014-11-15
% $timetable:
%   g10r2:
%     2014-11-15:
%       2:

\section*{Геометрический разнобой}

% $authors:
% - Дарья Сербина

\begin{problems}

\item
Дан треугольник $ABC$.
На~продолжениях сторон $AB$ и~$CB$ за~точку~$B$ взяты соответственно точки
$C_1$ и~$A_1$ так, что $AC = A_1 C = A C_1$.
Докажите, что описанные окружности треугольников $A B A_1$ и $C B C_1$
пересекаются на~биссектрисе угла~$B$.

\item
Три окружности попарно пересекаются в~точках
$A_1$ и~$A_2$, $B_1$ и $B_2$, $C_1$ и $C_2$.
Докажите, что
$A_1 B_2 \cdot B_1 C_2 \cdot C_1 A_2 = A_2 B_1 \cdot B_2 C_1 \cdot C_2 A_1$.

\item
В~четырехугольнике $ABCD$ углы $A$ и~$C$~--- прямые.
На~сторонах $AB$ и~$CD$ как на~диаметрах построены окружности, пересекающиеся
в~точках $X$ и~$Y$.
Докажите, что прямая~$XY$ проходит через середину~$K$ диагонали~$AC$.

\item
Даны четыре точки $A$, $B$, $C$, $D$.
Известно, что любые две окружности, одна из~которых проходит через $A$ и~$B$,
а~другая~--- через $C$ и~$D$, пересекаются.
Докажите, что общие хорды всех таких пар окружностей проходят через одну точку.

\item
Внутри четырехугольника $ABCD$ взята точка~$M$ так, что $ABMD$~---
параллелограмм.
Докажите, что если $\angle CBM = \angle CDM$, то $ \angle ACD = \angle BCM$.

\end{problems}

