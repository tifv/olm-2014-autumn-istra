% $date: 2014-11-15
% $timetable:
%   g10r1:
%     2014-11-15:
%       1:

\section*{Основания биссектрис}

% $authors:
% - Алексей Доледенок

Дан треугольник $ABC$;
$O$~--- центр описанной окружности;
$I$~--- центр вписанной окружности;
$I_A$, $I_B$ и~$I_C$~--- центры вневписанных окружностей;
$A_1$, $B_1$ и~$C_1$~--- основания биссектрис;
$A_0$, $B_0$ и~$C_0$~--- середины дуг $BC$, $AC$, $AB$.

\begin{problems}

\item
Докажите, что $B_1 C_1$~--- радикальная ось описанной окружности треугольников
$ABC$ и~$I_B I I_C$. 

\item
Пусть $D$ и~$E$~--- точки пересечения прямой~$B_1 C_1$ с~описанной окружностью
треугольника $ABC$.
Докажите, что радиус описанной окружности треугольника $DIE$ в~два раза больше
радиуса описанной окружности треугольника $ABC$.

\item
Докажите, что прямые $B_1 C_1$ и~$O I_A$ перпендикулярны.

\item
Докажите, что основания внешних биссектрис лежат на~одной прямой,
перпендикулярной прямой~$OI$.

\item
Пусть $K$~--- основание внешней биссектрисы на~прямой~$AC$, $B'_0$~--- середина
дуги~$AC$, содержащая точку~$B$.
Докажите, что прямые $KI$ и~$I_B B'_0$ перпендикулярны.

\item
Пусть прямые $B_1 C_1$ и~$B_0 C_0$ пересекаются в~точке~$P_A$.
Докажите, что тогда $P_A I$ параллельно $BC$, а~$P_A A$ является касательной
к~описанной окружности треугольника $ABC$.

\item
Определим точки $P_B$ и~$P_C$ аналогично.
Докажите, что $P_A$, $P_B$ и~$P_C$ лежат на~одной прямой, которая параллельна
прямой, проходящей через основания внешних биссектрис.

\item
Пусть $L$~--- точка пересечения $B_0 C_1$ и~$B_1 C_0$.
Докажите, что прямая~$LI$ делит отрезок~$BC$ пополам.

\item
Описанная окружность треугольника $A_1 B_1 C_1$ высекает три отрезка
на~сторонах треугольника $ABC$.
Докажите, что один из~них равен сумме двух других.

\end{problems}

