% $date: 2014-11-14
% $timetable:
%   g10r2:
%     2014-11-14:
%       3:

\section*{Ещё немного об Эйлере}

% $authors:
% - Алексей Доледенок

Дан треугольник $ABC$;
$H$~--- его ортоцентр;
$M$~--- точка пересечения медиан;
$O$~--- центр описанной окружности;
$I$~--- центр вписанной окружности;
$M_A$, $M_B$, $M_C$~--- середины сторон $BC$, $AC$, $AB$ соответственно;
$H_A$, $H_B$, $H_C$~--- основания высот из~вершин $A$, $B$, $C$ соответственно;
$I_A$, $I_B$, $I_C$~--- центры вневписанных окружностей, касающихся сторон
$BC$, $AC$ и~$AB$ соответственно.

\begin{problems}

\item
Докажите, что отрезки, соединяющие $M_A$, $M_B$, $M_C$ с~серединами
$AH$, $BH$, $CH$ пересекаются в~одной точке.

\item
$P$~--- точка пересечения описанных окружностей треугольников $ABC$
и~$A H_B H_C$.
Докажите, что прямая~$PH$ делит отрезок~$BC$ пополам.

\item
Докажите, что одна из~дуг $H_A M_A$, $H_B M_B$, $H_C M_C$ окружности Эйлера
равна сумме двух других.

\item
\sp
Пусть $O_A$, $O_B$, $O_C$~--- центры описанных окружностей треугольников
$BCH$, $ACH$, $ABH$ соответственно.
Докажите, что треугольники $O_A O_B O_C$ и~$ABC$ равны.
\\
\sp
Докажите, что окружности Эйлера треугольников $ABC$ и $O_A O_B O_C$ совпадают.

\item
\sp
Докажите, что прямые, проходящие через $I_A$, $I_B$, $I_C$ перпендикулярно
$BC$, $AC$ и~$AB$ пересекаются в~одной точке~$X$.
\\
\sp
Докажите, что $X$, $O$, $I$ лежат на~одной прямой, причем $IO = OX$.

\item
Докажите, что прямая, соединяющая проекции $H$ на~биссектрисы внутреннего
и~внешнего углов $A$ делит отрезок~$BC$ пополам.

\item
\sp
В~треугольнике $ABC$ угол $\angle A$ равен $120^{\circ}$.
Докажите, что $OH = BA + AC$.
\\
\sp
Пусть $T$~--- точка внутри треугольника $ABC$ такая, что стороны треугольника
видны из~нее под углами $120^{\circ}$ \emph{(точка Торричелли)}.
Докажите, что прямая Эйлера треугольника $BCT$ параллельна $AT$.
\\
\sp
Докажите, что прямые Эйлера треугольников $ABC$, $ABT$, $ACT$, $BCT$
пересекаются в~одной точке.

\end{problems}

