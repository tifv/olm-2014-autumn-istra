% $date: 2014-11-14
% $timetable:
%   g11r2:
%     2014-11-14:
%       1:
%   g11r1:
%     2014-11-14:
%       2:

\section*{Стереометрия}

% $authors:
% - Фёдор Нилов

\begin{problems}

\item
Тетраэдр называется ортоцентрическим, если его высоты (или их~продолжения)
пересекаются в~одной точке.
Докажите, что ортоцентрическом тетраэдре общие перпендикуляры каждой пары
противоположных ребер пересекаются в~одной точке.

\item
В~четырехугольную пирамиду $ABCDS$, основанием которой является параллелограмм
$ABCD$, вписана сфера.
Докажите, что сумма площадей треугольников $ABS$ и~$CDS$ равна сумме площадей
треугольников $BCS$ и~$ADS$.

\item
Докажите, что если у~тетраэдра два отрезка, идущие из~концов некоторого ребра
в~центры вписанных окружностей противолежащих граней, пересекаются, то~отрезки,
выпущенные из~концов скрещивающегося с~ним ребра в~центры вписанных окружностей
двух других граней, также пересекаются.

\item
В~тетраэдре $ABCD$ из~вершины~$A$ опустили перпендикуляры $AB'$, $AC'$, $AD'$
на~плоскости, делящие двугранные углы при ребрах $CD$, $BD$, $BC$ пополам.
Докажите, что плоскость $(B'C'D')$ параллельна плоскости $(BCD)$.

\item
Высота четырехугольной пирамиды $SABCD$ проходит через точку пересечения
диагоналей её~основания $ABCD$.
Из~вершин основания опущены перпендикуляры $A A_1$, $B B_1$, $C C_1$, $D D_1$
на~прямые $SC$, $SD$, $SA$ и~$SB$ соответственно.
Оказалось, что точки $S$, $A_1$, $B_1$, $C_1$, $D_1$ различны и~лежат на~одной
сфере.
Докажите, что прямые $A A_1$, $B B_1$, $C C_1$, $D D_1$ проходят через одну
точку.

\item
Сфера $\omega$ проходит через вершину~$S$ пирамиды $SABC$ и~пересекает ребра
$SA$, $SB$ и~$SC$ вторично в~точках $A_1$, $B_1$ и~$C_1$ соответственно.
Сфера $\Omega$, описанная около пирамиды $SABC$, пересекается с~$\omega$
по~окружности, лежащей в~плоскости, параллельной плоскости $(ABC)$.
Точки $A_2$, $B_2$ и~$C_2$ симметричны точкам $A_1$, $B_1$ и~$C_1$ относительно
середин ребер $SA$, $SB$ и~$SC$ соответственно.
Докажите, что точки $A$, $B$, $C$, $A_2$, $B_2$ и~$C_2$ лежат на~одной сфере.

\item
B~основании четырехугольной пирамиды $SABCD$ лежит четырехугольник $ABCD$,
диагонали которого перпендикулярны и~пересекаются в~точке~$P$, и~$SP$ является
высотой пирамиды.
Докажите, что проекции точки~$P$ на~боковые грани пирамиды лежат на~одной
окружности.

\item
Плоскость $\alpha$ пересекает ребра $AB$, $BC$, $CD$ и~$DA$ треугольной
пирамиды $ABCD$ в~точках $K$, $L$, $M$ и~$N$ соответственно.
Оказалось, что двугранные углы
$\angle{(KLA, KLM)}$, $\angle{(LMB, LMN)}$, $\angle{(MNC, MNK)}$
и~$\angle{(NKD, NKL)}$ равны.
Докажите, что проекции вершин $A$, $B$, $C$ и~$D$ на~плоскость~$\alpha$ лежат
на~одной окружности.

\item
На~ребрах произвольного тетраэдра выбрано по~точке.
Через каждую тройку точек, лежащих на~ребрах с~общей вершиной, проведена
плоскость.
Докажите, что если три из~четырех проведенных плоскостей касаются вписанного
в~тетраэдр шара, то~и~четвертая плоскость также его касается.

\end{problems}

