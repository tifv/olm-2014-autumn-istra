% $date: 2014-11-17
% $timetable:
%   g9r2:
%     2014-11-17:
%       1:

\section*{Радикальная ось и степень точки}

% $authors:
% - Фёдор Ивлев

% Рассказать:
% - определение того, что такое степень точки.
% - связь с секущими и отрезками хорд.
% - сказать что-то про касательную.
% - что такое радикальная ось двух окружностей.
% - почему радикальная ось двух пересекающихся окружностей -- это прямая,
%   проходящая через точки пересечения.
% Напомнить, что точка --- это тоже окружность, просто нулевого радиуса. Для
% неё тоже можно посчитать степень точки, а следовательно считать
% и радикальные оси.
% Рассказать, как строится отрезок равный корню из произведения.
% Разобрать задачу из разнобоя.
% Показать как степень точки можно применить к произведениям связанным
% с высотами

\begin{problems}

\item
\sp
Докажите, что если центры трех окружностей не~лежат на~одной прямой,
то~существует единственная точка плоскости, имеющая одинаковую степень точки
относительно всех трёх окружностей.
Эта точка называется \textbf{радикальным центром} этих трёх окружностей.
\\
\sp
Проведено три попарно пересекающиеся окружности.
Докажите, что их~общие хорды (или прямые, их~содержащие) пересекаются в~одной
точке.

\item
\sp
Докажите, что середины четырех общих касательных к~двум непересекающимся кругам
коллинеарны.
\\
\sp
Через две из~точек касания общих внешних касательных с~двумя окружностями
проведена прямая.
Докажите, что окружности высекают на~этой прямой равные хорды.

\item
Точки $A_1$ и~$A_2$ лежат на~стороне~$BC$, $B_1$ и~$B_2$~--- на~стороне~$AC$,
$C_1$ и~$C_2$~--- на~стороне~$AB$.
Известно, что точки
$A_1$, $A_2$, $B_1$, $B_2$ лежат на~одной окружности;
$B_1$, $B_2$, $C_1$, $C_2$ лежат на~одной окружности;
$C_1$, $C_2$, $A_1$, $A_2$ лежат на~одной окружности.
Докажите, что все шесть точек лежат на~одной окружности.

\item
На~сторонах $AC$, $AB$ треугольника $ABC$ отмечены точки $B_1$, $C_1$
соответственно.
На~отрезках $B B_1$, $C C_1$ как на~диаметрах построили окружности.
Докажите, что прямая, проходящая через точки пересечения окружностей, содержит
ортоцентр треугольника.

\item
Пусть $I$~--- центр вписанной окружности треугольника $ABC$.
Перпендикуляр, восстановленный в~точке~$I$ к~отрезку~$AI$, пересекает
прямую~$BC$ в~точке~$P$.
Точка~$Q$~--- основание перпендикуляра, опущенного из~$I$ на~$AP$.
Докажите, что $Q$ лежит на~описанной окружности треугольника $ABC$.

\item
Прямая~$OA$ касается окружности в~точке~$A$, а~хорда~$BC$ параллельна~$OA$.
Прямые $OB$ и~$OC$ вторично пересекают окружность в~точках $K$ и~$L$.
Докажите, что прямая~$KL$ делит отрезок~$OA$ пополам.

\item
Постройте окружность, проходящую через две заданные точки и~касающуюся данной
прямой.

\item
Точка~$M$~--- середина хорды~$AB$.
Хорда~$CD$ пересекает~$AB$ в~точке~$M$.
На~отрезке~$CD$ как на~диаметре построена полуокружность.
Точка~$E$ лежит на~этой полуокружности, и~$ME$~--- перпендикуляр к~$CD$.
Найдите угол $\angle AEB$.

\item
В~обозначениях предыдущей задачи добавим точку~$P$ пересечения касательных,
проведённых к~окружности в~точках $A$ и~$B$.
Докажите, что $\angle CPM = \angle DPM$.

%\item
%Две окружности пересекаются в~точках $E$ и~$F$.
%Прямая~$l$ пересекает первую окружность в~точках $A$ и~$B$, вторую~--- в~точках
%$C$ и~$D$ так, что точка~$E$ лежит внутри треугольника $ADF$, а~точки~$B$
%и~$C$~--- на~отрезке~$AD$.
%Оказалось, что $AB = CD$.
%Докажите, что $BE \cdot DF = CE \cdot AF$.
%% Питерская городская олимпиада, первый тур, 2011.10.5 (Д. Максимов)

\end{problems}

