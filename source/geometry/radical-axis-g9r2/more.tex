% $date: 2014-11-19
% $timetable:
%   g9r2:
%     2014-11-19:
%       2:

\section*{Радикальные оси~--- 2}

% $authors:
% - Фёдор Ивлев

\begin{problems}

%\item
%Обозначим основания высот неравнобедренного треугольника $ABC$, проведённых
%из~точек $A$, $B$ и~$C$, через $A_1$, $B_1$ и~$C_1$ соответственно.
%Прямые $A_1 B_1$ и~$AB$ пересекаются в~точке~$P$, а~прямые~$A_1 C_1$ и~$AC$~---
%в~точке~$Q$.
%Докажите, что $PQ \perp OM$, где $O$ и~$M$~--- центр описанной окружности
%и~точка пересечения медиан треугольника $ABC$ соответственно.

\item
На~плоскости даны окружность~$\omega$, точка~$A$, лежащая внутри $\omega$,
и~точка~$B$ ($B \neq A$).
Рассматриваются всевозможные треугольники $BXY$ такие, что точки $X$ и~$Y$
лежат на~$\omega$ и~хорда~$XY$ проходит через точку~$A$.
Докажите, что центры окружностей, описанных около треугольников $BXY$, лежат
на~одной прямой.
% (П. Кожевников) (степень точки) ВМО, окружной этап, 99.4.10.2

\item
Вписанная окружность~$\gamma$ треугольника $ABC$ касается его стороны~$BC$
в~точке~$D$.
Обозначим центр вневписанной в~угол~$A$ окружности через~$I$, а~середину
отрезка~$DI$ через~$M$.
Докажите, что отрезок касательной, проведённой из~точки~$M$ к~$\gamma$
равен~$MB$.

\item
Пусть $AD$~--- биссектриса угла~$A$ треугольника $ABC$.
Окружность~$\omega$ касается стороны~$BC$ в~точке~$D$ и~проходит через
точку~$A$.
Обозначим точку пересечения~$\omega$ с~$AB$, отличную от~$A$, через~$P$.
Прямая~$PC$ вторично пересекает~$\omega$ в~точке~$Q$.
Докажите, что $AQ$ делит отрезок~$DC$ пополам.

%\item
%Серединный перпендикуляр к~стороне~$AC$ треугольника $ABC$ пересекает прямые
%$BA$ и~$BC$ в~точках $B_1$ и~$B_2$.
%Серединный перпендикуляр к~стороне $AB$ пересекает прямые $CA$ и~$CB$ в~точках
%$C_1$ и~$C_2$.
%Описанные окружности треугольников $C C_1 C_2$ и~$B B_1 B_2$ пересекаются
%в~точках $P$ и~$Q$.
%Докажите, что центр описанной окружности треугольника $ABC$ лежит
%на~прямой~$PQ$.

\item
Пусть $A_1$, $B_1$, $C_1$~--- точки касания вписанной в~треугольник $ABC$
окружности со~сторонами $BC$, $CA$, $AB$ соответственно.
Точка~$P$~--- произвольная.
Серединный перпендикуляр к~отрезку~$P A_1$ пересекает прямую~$BC$
в~точке~$A_2$.
Точки $B_2$, $C_2$ определяются аналогично.
Докажите, что $A_2$, $B_2$, $C_2$ лежат на~одной прямой.

%\item
%Пусть $B_1$, $C_1$~--- точки касания вписанной окружности треугольника $ABC$
%со~сторонами $AC$ и~$AB$.
%На~продолжениях сторон $AB$, $AC$ за~точки $B$ и~$C$ отметили точки $X$, $Y$
%соответственно так, что $C_1 X = B_1 Y = BC$.
%Докажите, что середины отрезков $C_1 X$, $B_1 Y$, $BC$ лежат на~одной прямой.

\end{problems}

