% $date: 2015-11-13
% $timetable:
%   g11r2:
%     2014-11-13:
%       3:

% $caption: Решётки (алгебра)

\section*{Решётки}

% $authors:
% - Андрей Кушнир

\begingroup
    \def\abs#1{\lvert #1 \rvert}%

\claim{Бесполезные определения}
Множество~$G$ векторов на~прямой / в~плоскости / в~пространстве будем называть
\emph{группой,} если оно содержит нулевой вектор и~замкнуто относительно
операций сложения векторов и~взятия противоположного вектора.
Если кто не~понял, имеется ввиду, что если $x \in G$ и~$y \in G$, то~тогда
$0 \in G$, $x + y \in G$ и~$-x \in G$.
Подмножество группы, само являющееся группой, будем называть \emph{подгруппой.}

\begin{problems}

\item
В~трех вершинах квадрата сидят кузнечики.
Каждый год один из~кузнечиков перепрыгивает через одного из~других
(т.~е. отражается центрально-симметрично).
В~конце они вновь оказываются в~каких-то трех вершинах исходного квадрата.
Докажите, что каждый кузнечик сидит в~своей стартовой вершине.

\item
На~плоскости расположена фигура площади больше $1$
(фигура~--- это объединение конечного числа многоугольников).
Докажите, что в~ней можно отметить две различные точки так, чтобы вектор, их
соединяющий, имел целые координаты.

\item
\emph{Целые гауссовы числа}~--- комплексные числа вида $a + bi$, где
$a$, $b$~--- целые.
Докажите, что если $z$ и~$w$ целые гауссовы, $w \neq 0$, то~существуют
целые гауссовы $s$, $r$ такие, что $z = w s + r$ и~$\abs{r} < \abs{w}$.
Таким образом целые гауссовы числа можно делить друг на~друга с~остатком.
Такие $r$ и~$w$ не~обязательно определены однозначно.

\item
\label{algebra/grid/main:problem:basis-volume}%
Два вектора с~целыми координатами таковы, что все целые точки плоскости можно
выразить их целочисленными линейными комбинациями.
Докажите, что площадь параллелограмма, натянутого на~эти два вектора,
равна $1$.

\item
Кузнечики живут ровно по~$100$ лет.
Каждый кузнечик рождается ровно в~момент начала какого-то года.
В~какой-то момент появился первый кузнечик, в~какой-то момент умрет последний.
За~всю историю случилось нечетное число кузнечиков.
Докажите, что существует по~крайней мере $100$ лет (возможно, не~подряд),
когда существовало нечетное число кузнечиков.

\item
Внутри правильного треугольника расположена точка.
Ее поотражали несколько раз относительно прямых, содержащих стороны,
и она опять попала внутрь треугольника.
Докажите, что она вернулась на~исходное место.

\item
На~плоскости отмечены некоторые целые точки.
В~любом круге радиуса $2014$ хотя~бы одна есть.
Докажите, что существуют четыре отмеченные точки, лежащие на~одной окружности.

\end{problems}

\claim{Ещё бесполезное определение}
\emph{Дискретная} группа векторов ~--- это когда в~некотором шарике с~центром
в~начале координат нет концов векторов, кроме нулевого
(все векторы торчат из~начала координат).

\begin{problems}

\item\emph{(Теорема о~решетках)}
\label{algebra/grid/main:problem:basis-exists}%
\sp На~прямой;
\quad
\sp на~плоскости;
\quad
\sp в~пространстве
\\
дана дискретная группа векторов (смотри бесполезные определения).
Докажите, что можно выбрать несколько векторов (не~больше чем размерность
пространства, возможно нуль), что любой элемент группы представляется как
целочисленная линейная комбинация выбранных, причем единственным образом.

\end{problems}

Дискретная группа векторов в~пространстве называется \emph{решеткой.}
Задача~\ref{algebra/grid/main:problem:basis-exists} объясняет, почему именно.
Набор векторов, который мы в~ней строим, называется \emph{базисом решетки.}
Задача~\ref{algebra/grid/main:problem:basis-volume} говорит, что если в~решетке
выбраны два базиса, то~площади параллелограммов, натянутых на~эти базисы, равны.

\endgroup % \def\abs

