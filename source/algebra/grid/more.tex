% $date: 2015-11-14
% $timetable:
%   g11r2:
%     2014-11-14:
%       3:

% $matter[-contained,no-header]:
% - verbatim: \section*{Решётки}
% - .[contained]

% $caption: Решётки (алгебра). Ещё две задачки

\subsection*{Ещё две задачки}

% $authors:
% - Андрей Кушнир

\begin{problems}

\item
Рассмотрим на~плоскости множество точек с~вещественными координатами $(x, y)$
таких, что неравенство
\[
    m x + n y
\geq
    \frac{m^2 + n^2}{2}
\]
имеет ровно $2014$ целых решений $(m, n)$.
Найдите площадь этого множества.

\item
Через точки с~целыми координатами в~пространстве провели все плоскости,
параллельные координатным.
Пространство разбилось на~кубики.
$a$, $b$, $c$~--- натуральные числа, взаимно простые в~совокупности.
Кубики высекают на~плоскости $ax + by + cz = 0$ многоугольники.
Многоугольники на~плоскости называются \emph{эквивалентными,} если они
получаются друг из~друга параллельным переносом.
Докажите, что число классов эквивалентности не~превосходит $a + b + c$.

\end{problems}

