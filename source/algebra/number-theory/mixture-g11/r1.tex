% $date: 2014-11-11
% $timetable:
%   g11r1:
%     2014-11-11:
%       3:

\section*{Теория чисел}

% $authors:
% - Олег Орлов

\begingroup
    \def\abs#1{\lvert #1 \rvert}

\begin{problems}

\item
Пусть $a$ и~$b$~--- различные натуральные числа такие, что $a b (a + b)$
делится на~$a^2 + a b + b^2$.
Докажите, что $\abs{a - b} > \sqrt[3]{ab}$.

\item
Через $S(n)$ обозначим сумму цифр числа~$n$.
Докажите, что не~существует натурального~$N$ такого, что для любого $n > N$
выполнялось~бы неравенство $S(2^n) \leq S(2^{n+1})$.

\item
Найдите все такие натуральные $n \geq 2$, что числа $C_{n-k}^k$~--- четные при
$k = 1, 2, \ldots, [n / 2]$. 

\item
Докажите, что при любом простом нечетном~$p$ число $p^{p+1} + (p + 1)^p$
не~является полным квадратом.

%\item
%Для любого многочлена $P(x)$ с~целыми коэффициентами обозначим количество его
%нечетных коэффициентов через $\omega(P)$.
%Обозначим через $Q_i(x) = (1+x)^i$.
%Докажите, что для любых целых $0 \leq i_1 < i_2 < \ldots < i_n$ выполнено
%неравенство $\omega(Q_{i_1} + Q_{i_2} +  \ldots + Q_{i_n}) \geq
%\omega(Q_{i_1})$.

\item
Последовательность $\{a_n\}_{n \geq 0}$ определяется следующим образом:
$a_0$~--- положительное рациональное число меньшее $\sqrt{1998}$, и~если
$a_n = p_n / q_n$, где $p_n$ и~$q_n$~--- взаимно просты,
то~$a_{n+1} = (p_n^2 + 5) / (p_n q_n)$.
Докажите, что $a_n < \sqrt{1998}$ для любого натурального $n$.

\item
Известно, что $a$, $b$, $c$, $m$~--- натуральные числа такие, что
$1 + a^2 + b^2 + c^2 = a b c m$.
Докажите, что $m = 4$.

%\item
%При каких натуральных $n$ найдутся такие положительные рациональные,
%но~не~целые числа $a$ и~$b$, что оба числа $a+b$ и~$a^n + b^n$ -- целые?

\end{problems}

\endgroup % \def\abs

