% $date: 2014-11-11
% $timetable:
%   g11r2:
%     2014-11-11:
%       1:

\section*{Теория чисел}

% $authors:
% - Олег Орлов

\begingroup
    \def\abs#1{\lvert #1 \rvert}

\begin{problems}

\item
Для каждого натурального~$n$ найдите наибольший общий делитель чисел $n! + 1$
и~$(n + 1)!$.

\item
Дано натуральное число~$n$.
Докажите, что существует число, состоящее только из~нулей и~единиц, делящееся
на~$n$.

\item
$a$ и~$b$~-- различные натуральные числа такие, что $a b (a + b)$ делится
на~$a^2 + a b + b^2$.
Докажите, что $\abs{a - b} > \sqrt[3]{a b}$.

\item
Через $S(n)$ обозначим сумму цифр числа~$n$.
Докажите, что не~существует натурального~$N$ такого, что для любого $n > N$
выполнялось~бы неравенство $S(2^n) \leq S(2^{n+1})$.

\item
Докажите, что при любом натуральном~$n$ число $2^n - 1$ не~делится на~$n$.

\item
Докажите, что при любом простом нечетном~$p$ число $p^{p+1} + (p + 1)^p$
не~является полным квадратом. 

%\item
%При каких натуральных $n$ найдутся такие положительные рациональные,
%но~не~целые числа $a$ и~$b$, что оба числа $a + b$ и~$a^n + b^n$~--- целые?

\end{problems}

\endgroup % \def\abs

