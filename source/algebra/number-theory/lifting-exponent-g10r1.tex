% $date: 2014-11-11
% $timetable:
%   g10r1:
%     2014-11-11:
%       1:

\section*{Лемма об уточнении показателя}

% $authors:
% - Андрей Меньщиков

\begingroup
    \def\ord{\operatorname{ord}}

\definition
\emph{Порядком вхождения $\ord_p (n)$} будем обозначать степень, в~которой
простое число~$p$ входит в~разложение~$n$ на~простые множители.

\claim{Лемма об уточнении показателя%
\footnote{англ. lifting the exponent lemma}}
Даны простое число~$p$ и~натуральные числа $k$, $a$ и~$b$, причем
$p \mid a - b$, $a \neq b$, $a$ и~$b$ не~делятся на~$p$.
Тогда если $p > 2$ или $\ord_p (a - b) > 1$,
то~$\ord_p (a^k - b^k) = \ord_p (a - b) + \ord_p(k)$.

\claim{Замечание}
Иногда эту лемму называют \emph{леммой Гензеля,} но~это не~совсем правильно:
это другое утверждение, хотя и~близкое по~смыслу.

\begin{problems}

\item
На~какую максимальную степень пятерки делится число $3^{1000} - 2^{1000}$?

\item
Сколькими нулями оканчивается число $4^{5^6} + 6^{5^4}$?

\item
При каких натуральных значениях~$n$ число $(2014^n - 1)$ делится
на~$(100^n - 1)$?

\item
Пусть натуральные числа $x$, $y$, $p$, $n$ и~$k$ таковы, что
$x^n + y^n = p^k$.
Докажите, что если число~$n$ ($n > 1$)~--- нечетное, а~число $p$~--- нечетное
простое, то~$n$ является степенью числа~$p$ (с~натуральным показателем).

\item
Решите в~натуральных числах уравнение $3^{x} = 2^{x} y + 1$.

\item
Докажите, что порядок числа~2 по~модулю $3^n$ (т.~е. наименьшее натуральное
число~$d$ такое, что $2^d \equiv 1 \pmod{3^{n}}$) равен $\phi(3^n)$.

\item
Решите в~натуральных числах уравнение $z^x + 1 = (z + 1)^y$.

\item
Пусть $a, b \in \mathbb{N}$.
Докажите, что лишь для конечного числа $n \in \mathbb{N}$ сумма
$\left(a + \frac{1}{2}\right)^n + \left(b + \frac{1}{2}\right)^n$~--- целая.

\end{problems}

\endgroup % \def\ord

