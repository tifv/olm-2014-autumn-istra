% $date: 2014-11-10
% $timetable:
%   g11r2:
%     2014-11-10:
%       2:

\section*{Биномиальные коэффициенты в теории чисел}

% $authors:
% - Олег Орлов

Считаем, что $C_n^k = 0$ при $n<k$.

\begin{problems}

\item
Пусть $n$~--- натуральное нечетное число.
Докажите, что среди чисел\[
    \{ C_n^1, C_n^2, \ldots, C_n^{\frac{n-1}{2}} \}
\]
нечетное число нечетных чисел.

\item
Пусть $p$~--- простое.
Докажите, что для любых натуральных $n$ и~$k$ число $C_{p^n}^k$ делится на~$p$.

\item
Пусть $m$, $n$~--- натуральные числа, $1 \leq m \leq n$.
Докажите, что $m$ является делителем числа
\[
    n (C_n^0 - C_n^1 + C_n^2 - \ldots + (-1)^{m-1}C_n^{m-1})
\,.\]

\item\emph{Биномиальная система счисления.}
Пусть $k$~--- некоторое натуральное число.
Докажите, что любое натуральное число $n$ можно единственным образом
представить в~виде:
\[
    n = C_{a_1}^1 + C_{a_2}^2 + \ldots + C_{a_k}^k
\,,\]
где $0 \leq a_1 < a_2 < a_3 < \ldots < a_k$.

\item
Докажите, что для любого натурального числа $n \geq 3$ верно неравенство
\[
    \text{НОК}(1, 2, \ldots, n) > 2^{n-1}
\,.\]

\item
Докажите, что для любого натурального числа $n$ сумма
\[
    \sum\limits_{k=0}^n
        C_{2n+1}^{2k+1} 2^{3k}
\]
не~делится на~$5$.

\end{problems}

