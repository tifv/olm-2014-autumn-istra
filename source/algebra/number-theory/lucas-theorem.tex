% $date: 2014-11-10
% $timetable:
%   g11r1:
%     2014-11-10:
%       3:

\setcounter{footnote}{0}

% $caption: Теорема Люка

\section*{Биномиальные коэффициенты в~теории чисел.\\Теорема Люка}

% $authors:
% - Олег Орлов

\begingroup
    \def\ov{\overline}

Считаем, что $C_0^0 = 1$ и~$C_n^k = 0$ при $n < k$.

\begin{problems}

\item
Пусть $m$, $n$~--- натуральные числа, $1 \leq m \leq n$.
Докажите, что $m$ является делителем числа
\[
    n (C_n^0 - C_n^1 + C_n^2 - \ldots + (-1)^{m-1}C_n^{m-1})
\, . \]

\item\emph{Биномиальная система счисления.}
Пусть $k$~--- некоторое натуральное число.
Докажите, что любое натуральное число $n$ можно единственным образом
представить в~виде:
\[
    n = C_{a_1}^1 + C_{a_2}^2 + \ldots + C_{a_k}^k
\, , \]
где $0 \leq a_1 < a_2 < a_3 < \ldots < a_k$.

\item
Докажите, что для любого натурального числа $n \geq 3$ верно неравенство
\[
    \text{НОК}(1, 2, \ldots, n) > 2^{n-1}
\, . \]

\item
Докажите, что для любого натурального $n$ число
\[
    S_n
=
    C_{2n+1}^0 \cdot 2^{2n}
    +
    C_{2n+1}^2 \cdot 2^{2n-2} \cdot 3
    + \ldots +
    C_{2n+1}^{2n}\cdot 3^n
\]
является суммой двух подряд идущих квадратов натуральных чисел.

\end{problems}

\theoremof{Люк\'{а}}
%\begingroup\let\thefootnote\relax\footnotetext[0]{%
%François Édouard Anatole Lucas (1842--1891)}\endgroup
Пусть $p$~--- простое, $n$~--- натуральное
и~$n = \ov{n_m n_{m-1}  \ldots n_0}_p$
(т.~е. $n = n_0 + n_1 p + \ldots + n_m p^m$, $n_m \neq 0$).
Также пусть дано натуральное число $i < n$.
Тогда, если $i = i_0 + i_1 p + \ldots + i_m p^m$, где
$0 \leq i_0, i_1, \ldots, i_m \leq p-1$,
то~\[
    C_n^i
\equiv
    C_{n_0}^{i_0} \cdot C_{n_1}^{i_1} \cdot \ldots \cdot C_{n_m}^{i_m}
\pmod{p}
\, . \]

\begin{problems}

\item
Будем говорить, что многочлены $f(x)$ и~$g(x)$ \emph{сравнимы по~модулю~$p$,}
если все коэффициенты многочлена $\bigl(f(x) - g(x)\bigr)$ делятся на~$p$
(обозначать будем $f(x) \equiv g(x) \pmod{p}$, помня при этом, что $f(x)$
и~$g(x)$~--- многочлены).
\\
\subproblem
Пусть $p$~--- простое.
Для любого натурального $n$ докажите, что
\[
    (1 + x)^{p^n} \equiv 1 + x^{p^n} \pmod{p}
\, . \]
\subproblem
Докажите теорему Лукаса.
	
\item
Пусть $n$~--- натуральное число.
Докажите, что количество чисел $k \in \{0, 1, 2, \ldots, n\}$, таких, что
$C_n^k$~--- нечетное, является степенью двойки.

\item
Пусть $p$~--- простое нечетное число.
Найдите все натуральные числа $n$ такие, что каждое из~чисел
$C_n^1, C_n^2, \ldots, C_n^{n-1}$ делится на~$p$.

\item
Докажите, что все числа вида $C_{2^n}^k$, $k = 1, 2, \ldots, 2^n - 1$ являются
четными, и только ровно одно из них не делится на $4$.

\end{problems}

\endgroup % \dev\ov

