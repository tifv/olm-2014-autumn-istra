% $date: 2014-11-12
% $timetable:
%   g9r2:
%     2014-11-12:
%       2:
%   g9r1:
%     2014-11-12:
%       3:

\section*{Теорема Безу. Вокруг да около}

% $authors:
% - Владимир Трушков

\begin{problems}

\item
Найдите остаток от~деления многочлена $x^5 - 17 x + 1$ на~$x + 2$.

\item
Найдите остатки от~деления многочлена $x^{81} + x^{27} + x^9 + x^3 + 1$
на~$(x - 1)$ и~$(x^2 - 1)$.

\item
Многочлен $P(x)$ при делении на~$(x - 1)$ дает остаток $2$, а~при делении
на~$(x - 2)$ дает остаток $1$.
Какой остаток дает $P(x)$ при делении на~$(x - 1) (x - 2)$?

\item
Многочлен $P(x)$ при делении на~$x^2 - 4$ дает остаток $x + 1$,
а~на~$x^2 - 1$~--- остаток $x + 2$.
Найдите остаток при делении $P(x)$ на~$(x^2 - 4) (x^2 - 1)$.

\item
Докажите, что если значения двух многочленов, степени которых
не~превосходят $n$, совпадают в~$n + 1$ различных точках, то~эти многочлены
равны.

\item
Дан многочлен $P(x)$ такой, что многочлен $P(x^n)$ делится на~$(x - 1)$.
Докажите, что многочлен $P(x)$ также делится на~$(x - 1)$.

\item
Известно, что многочлен $x^n + x + 1$ делится на~$x^2 + x + 1$.
Докажите, что $n$ есть число вида $3 k + 2$.

\item
Многочлен $P(x^3) + Q(x^3)$ делится на~многочлен $x^2 + x + 1$.
Докажите, что многочлен $P(x) + Q(x)$ делится на~многочлен $x - 1$.

\item
При каких $n$ многочлен $1 + x^2 + x^4 + \ldots + x^{2n-2}$ делится
на~$1 + x + x^2 + \ldots + x^{n-1}$?

\end{problems}

