% $date: 2014-11-11
% $timetable:
%   g10r1:
%     2014-11-11:
%       3:

\section*{Производная многочлена}

% $authors:
% - Андрей Меньщиков

\begin{problems}

\item
Докажите, что произвольный непостоянный многочлен с~действительными
коэффициентами начиная с~некоторого момента монотонно возрастает или убывает.

\item
Пусть $P(x)$~--- многочлен с~целыми коэффициентами.
Докажите, что все коэффициенты его $n$-й производной $P^{(n)}(x)$ делятся
на~$n!$ для любого натурального~$n$.

%\item
%Можно~ли миллионом внутренностей парабол покрыть плоскость?
%(Параболы можно поворачивать и~разворачивать.)

\item
Докажите, что многочлен $P(x)$ степени $n > 1$ имеет кратный корень тогда
и~только тогда, когда $P(x)$ и~$P'(x)$ имеют общий корень.

\item
Докажите, что многочлен
\[
    P(x)
=
    1 + \frac{x^1}{1!} + \frac{x^2}{2!} + \ldots + \frac{x^n}{n!}
\]
не~имеет кратных корней.

\item
У~многочлена степени~$n$ имеется $n$~действительных корней.
Докажите, что их~среднее арифметическое равно среднему арифметическому корней
производной этого многочлена.

\item
Четыре корня многочлена четвертой степени образуют арифметическую прогрессию.
Докажите, что корни его производной также образуют арифметическую прогрессию.

\item
Исходно на~доске написаны многочлены $x^2 - 4 x$ и~$x^3 - 3 x^2 + 5$.
Если на~доске написаны многочлены $f(x)$ и~$g(x)$, то~разрешается дописать
на~нее многочлены $f(x) \pm g(x)$, $f(x) g(x)$, $f(g(x))$
и~$\lambda f(x)$, где $\lambda$ -- произвольная константа.
Может~ли на~доске после нескольких операций появиться многочлен вида $x^n-1$?

\item
Пусть $f$ и~$g$ -- многочлены степени $n$.
Докажите, что многочлен
\[
    f g^{(n)} - f' g^{(n-1)} + f'' g^{(n-2)} - \ldots + (-1)^n f^{(n)} g
\]
является константой.

\iffalse
%\item
%Пусть многочлен $f(x) = x^n + a x^{n-1} + b x^{n-2} + \ldots + a_0$ имеет
%$n$ различных действительных корней.
%Докажите, что $a^2 > \frac{2 b n}{n - 1}$.
\fi

\item
Можно~ли из~какой-либо точки плоскости провести к~графику многочлена $n$-й
степени более, чем $n$~касательных?

%\item
%Пусть $P(x) = (x - x_1) \cdot (x - x_2) \cdot \ldots \cdot (x - x_n)$,
%где $x_1, x_2, \ldots, x_n$~--- действительные числа.
%Докажите, что $(P'(x))^2 \geq P(x) \cdot P''(x)$ для всех действительных $x$.

%\item
%Докажите, что многочлен
%\(
%    P(x)
%=
%    a_0 + a_1 x^{k_1} + a_2 x^{k_2} + \ldots + a_n x^{k_n}
%\)
%имеет не~более $n$ положительных корней.

\item
Докажите, что при умножении многочлена $(x + 1)^{n-1}$ на~любой многочлен,
отличный от~нуля, получается многочлен, имеющий не~менее $n$ ненулевых
коэффициентов.

\item
У~многочлена степени~$n$ имеется $n$ различных корней.
Какое наибольшее число его коэффициентов может равняться нулю?

%\item
%Пусть многочлен $P(x)=a_nx^n+\ldots +a_0$ имеет хотя~бы один действительный
%корень и~$a_0\neq 0$.
%Докажите, что последовательно вычеркивая в~некотором порядке одночлены
%в~записи $P(x)$, можно получить из~него число $a_0$ так, чтобы каждый
%промежуточный многочлен также имел хотя~бы один действительный корень.

\end{problems}

