% $date: 2014-11-11
% $timetable:
%   g9r2:
%     2014-11-11:
%       2:
%   g9r1:
%     2014-11-11:
%       1:

\section*{Теорема Виета}

% $authors:
% - Владимир Трушков

\begingroup
    \def\ov{\overline}

\begin{problems}

\item
Дано уравнение $x^2 + p x + q = 0$.
Составьте с~помощью коэффициентов $p$ и~$q$ квадратное уравнение, корнями
которого были~бы $y_1 = x_1^2 + x_2^2$, $y_2 = x_1^3 + x_2^3$.

\item
При каких значениях параметра~$a$ множеством решений неравенства
$x^2 + a x - 1 < 0$ будет интервал длины~5?

\item
При каких значениях параметра~$a$ сумма квадратов корней уравнения
$4 x^2 - 28 x + a = 0$ равна $22.5$?

\item
При каком значении параметра~$a$ сумма квадратов корней уравнения
$x^2 + x \sqrt{a^2 - 4 a} - a - 2 = 0$ принимает наименьшее значение?

\item
При каких $a$ разность корней уравнения $2 x^2 - (a + 1) x + (a - 1) = 0$ равна
их~произведению?

\item
Квадратное уравнение $x^2 - 6 p x + q = 0$ имеет два различных корня $x_1$
и~$x_2$.
Числа $p$, $x_1$, $x_2$, $q$~--- четыре последовательных члена геометрической
прогрессии.
Найдите $x_1$ и~$x_2$.

\item
Определите все значения параметра~$a$, при каждом из~которых три различных
корня уравнения $x^3 + (a^2 - 9 a) x^2 + 8 a x - 64 = 0$ образуют
геометрическую прогрессию.
Найдите эти корни.

\item
При каких значениях $a$ четыре корня уравнения
$x^4 + (a - 5) x^2 + (a + 2)^2 = 0$ являются последовательными членами
арифметической прогрессии?

\item
Известно, что $\alpha$~--- корень уравнения $a x^2 + b x + b = 0$,
$\beta$~--- корень уравнения $a x^2 + a x + b = 0$, а~также, что
$\alpha \beta = 1$.
Найдите $\alpha$ и $\beta$.

\item
Известно, что каждое из~уравнений $x^2 + a x + b = 0$ и~$x^2 + b x + a = 0$
имеет два различных корня, и~эти четыре корня в~некотором порядке образуют
арифметическую прогрессию.
Найдите $a$ и~$b$.

\end{problems}

\endgroup % \dev\ov

