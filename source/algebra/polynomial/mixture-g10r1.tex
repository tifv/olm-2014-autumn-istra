% $date: 2014-11-10
% $timetable:
%   g10r1:
%     2014-11-10:
%       2:

\section*{Многочлены}

% $authors:
% - Андрей Меньщиков

\begin{problems}

\item
Существует~ли многочлен, который в~каждой положительной целой точке принимает
значение, равное ее~сумме цифр?

\item
Многочлен $P(x)$ дает остаток~5 при делении на~$(x - 2)$ и~остаток~7
при делении на~$(x - 3)$.
Какой остаток многочлен $P(x)$ дает при делении на~$(x - 2) (x - 3)$?

\item
Многочлен степени 101 таков, что в~каждой целой точке он~принимает целое
значение.
Какое наименьшее количество целых чисел может быть среди его коэффициентов?

\item
Существуют~ли два многочлена $P(x)$ и~$Q(x)$ с~целыми коэффициентами такие, что
множество значений рациональной функции $f(x) = P(x) / Q(x)$ есть
промежуток $[\sqrt{2}; +\infty)$?

\item
Многочлен $P(x)$ степени~$n$ таков, что
\[
    P(0) = 1
\,,\;
    P(1) = 1 / 2
\,,\;
    \ldots
,\;
    P(n) = 1 / (n + 1)
\,.\]
Найдите $P(2n)$.

\item
Даны взаимно простые многочлены $P(x)$ и~$Q(x)$ с~целыми коэффициентами.
Докажите, что существует натуральное число $C$ такое, что для любого целого~$n$
верно $(P(n), Q(n)) < C$.

\item
Найдите все натуральные $n > 1$ такие, что существуют различные целые числа
$x_1, \ldots, x_n$ и~многочлен с~целыми коэффициентами $P(x)$, для которых
\[
    P(x_1) = x_2
\,,\;
    P(x_2) = x_3
\,,\;
    \ldots
,\;
    P(x_n)=x_1
\, . \]

\item
Дан многочлен $P(x)$ с\quad
\subproblem натуральными
\quad
\subproblem целыми
\quad
коэффициентами и~положительным старшим коэффициентом.
Докажите, что найдется бесконечно много натуральных чисел $n_i$, таких, что
у~всех чисел $P(n_i)$ одинаковая сумма цифр.

\item
Найдите все многочлены $P(x)$ и~$Q(x)$ такие, что
$P(Q(x)) = (x - 1) \cdot (x - 2) \cdot \ldots \cdot (x - 35)$.

\item
Пусть $\alpha$ и~$\beta$~--- наибольшие действительные корни многочленов
$P(x) = 4 x^3 - 2 x^2 - 15 x + 9$ и~$Q(x) = 12 x^3 + 6 x^2 - 7 x + 1$
соответственно.
Докажите, что $\alpha^2 + 3 \beta^2 = 4$.

\end{problems}

