% $date: 2014-11-18
% $timetable:
%   g9r2:
%     2014-11-18:
%       1:
%       2:
%   g9r1:
%     2014-11-18:
%       1:
%       2:

\section*{Олимпиада 9 классов}

% $authors:
% - Владимир Брагин
% - Фёдор Ивлев

\begin{problems}

\item
Дан приведённый квадратный трехчлен $P(x)$.
Известно, что он~имеет общий корень с~многочленом $P(P(P(x)))$.
Докажите, что $P(0) \cdot P(1) = 0$.

\item
В~клетчатом квадрате $101 \times 101$ каждая клетка внутреннего квадрата
$99 \times 99$ покрашена в~один из~десяти цветов (клетки, примыкающие к~границе
квадрата, не~покрашены).
Может~ли оказаться, что в~каждом квадрате $3 \times 3$ в~цвет центральной
клетки покрашена еще ровно одна клетка?
% ВМО 2006.4.8.6

\item
В~остроугольном треугольнике $ABC$ проведены высоты $A A_1$, $B B_1$, $C C_1$.
Прямая, перпендикулярная стороне~$AC$ и~проходящая через точку~$A_1$,
пересекает прямую~$B_1 C_1$ в~точке~$D$.
Докажите, что угол~$ADC$ прямой.
% ВМО 2009.4.9.3

\item
В~компании из~семи человек любые шесть могут сесть за~круглый стол так, что
каждые два соседа окажутся знакомыми. 
Докажите, что и~всю компанию можно усадить за~круглый стол так, что каждые два
соседа окажутся знакомыми.
% ВМО 2010.3.9.7 Волчёнков С.Г.

\item
Неограниченная последовательность натуральных чисел $a_1, a_2, a_3, \ldots$
удовлетворяет следующему условию.
Существуют натуральное число $k$ и~такие ненулевые целые числа
$b_1, b_2, \ldots, b_k$, что для любого натурального $n$ выполнено равенство
$b_1 a_{n+1} + b_2 a_{n+2} + \ldots + b_k a_{n+k} = 0$.
Докажите, что в~последовательности $a_1, a_2, a_3, \ldots$ встретится хотя~бы
одно составное число.
% зимние сборы 2008.1.1

\end{problems}

