% $date: 2014-11-10
% $timetable:
%   g10r2:
%     2014-11-10:
%       2:

\section*{Таблички}

% $authors:
% - Юлий Тихонов

\begin{problems}

\itemy{0}
Дана таблица размером $8 \times 8$, изображающая шахматную доску.
За~каждый шаг разрешается поменять местами любые два столбца или любые две
строки.
Можно~ли за~нес\-колько шагов сделать так, чтобы верхняя половина таблицы стала
белой, а~нижняя половина~--- черной?

\itemy{0'}
В~угловой клетке таблицы $5 \times 5$ стоит плюс, а~в~остальных клетках стоят
минусы.
Разрешается в~любой строке или любом столбце поменять знаки на~противоположные.
Можно~ли за~несколько таких операций получить все знаки плюсами?

\item
В~клетках таблицы $2 N \times 2 N$ некоторым образом расставлены плюсы
и~минусы.
За~ход можно изменить знак во~всех клетках любого <<креста>>, то~есть
объединения некоторых строки и~столбца.
Докажите, что за~несколько ходов можно получить таблицу, во~всех клетках
которой стоят плюсы.

\item
В~клетки таблицы $m \times n$ вписаны некоторые числа.
Разрешается одновременно менять знак у~всех чисел некоторой строки или
некоторого столбца.
Докажите, что многократным повторением этой операции можно превратить данную
таблицу в~такую, у~которой суммы чисел, стоящих в~любой строке и~в~любом
столбце, неотрицательны.

\item
Во~всех клетках шахматной доски $8 \times 8$ расставлены плюсы, за~исключением
одной не~угловой клетки, где стоит минус.
Разрешается одновременно менять знак во~всех клетках одной горизонтали, одной
вертикали или одной диагонали
(диагональ~--- линия по~которой ходит шахматный слон).
Докажите, что такими операциями невозможно получить доску с~одними плюсами.

\item
В~каждой клетке таблицы $n \times n$ стоят минусы.
За~один ход можно поменять знаки в~одной фигуре $Z$-тетрамино
(то~есть клетчатой фигуре, которая получается сдвигом, поворотом или отражением
из~объединения клеток \texttt{a1}, \texttt{b1}, \texttt{b2}, \texttt{c2}
шахматной доски).
При каких $n$ можно получить таблицу со~всеми плюсами?

\item
Шахматная доска $6 \times 6$ покрыта $18$ костями домино $2 \times 1$.
Докажите, что при любом таком покрытии можно разрезать доску на~две части
по~горизонтальной или вертикальной линии, не~повредив ни~одной кости домино.

\item
Можно~ли расставить цифры $0$, $1$, $2$ в~клетках таблицы $100 \times 100$
таким образом, чтобы в~каждом прямоугольнике $3 \times 4$ клетки оказалось
$3$ нуля, $4$ единицы и~$5$ двоек?

\item
Какое наибольшее число дамок можно расставить на~шашечной доске $8 \times 8$
так, чтобы каждая дамка билась хотя~бы одной другой дамкой
(по~правилам игры в~шашки)?

\item
Какое наименьшее число фишек нужно поставить на~поля шахматной доски размером
\quad
\sp $8 \times 8$;
\quad
\sp $n \times n$
\quad
для того, чтобы на~каждой прямой, проходящей через центр произвольного поля
и~параллельной какой-либо стороне или диагонали доски, стояла хотя~бы одна
фишка?

\end{problems}

