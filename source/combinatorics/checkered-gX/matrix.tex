% $date: 2014-11-11
% $timetable:
%   gX:
%     2014-11-11:
%       1:

\section*{Клетчатая комбинаторика --- 2: матрицы}

% $authors:
% - Владимир Шарич

\begingroup
\providecommand\ifincludesolutions{\iffalse}

% $matter[with-solutions]:
% - verbatim: \begingroup\newcommand\ifincludesolutions{\iftrue}
% - .[-with-solutions]
% - verbatim: \endgroup

\begingroup
    \def\abs#1{\lvert #1 \rvert}

\emph{Задачи из short list'ов.}

\ifincludesolutions
\textbf{Версия с решениями.}
\emph{Решения являются слабообработанным переводом с~английского.
Beware!}
\fi

\begin{problems}

\item\textbf{(1997, IRN)}
Матрица $n \times n$, элементы которой принадлежат множеству целых чисел
от~1 до~$(2 n - 1)$, называется \emph{разнообразной}, если при всех
$i$ все $(2 n - 1)$ чисел в~$i$-той строке и~$i$-том столбце различны.
Докажите, что:
\\
\sp
не~существует разнообразной матрицы при $n = 1997$;
\\
\sp
разнообразные матрицы существуют для бесконечно многих $n$.

\end{problems}

\ifincludesolutions
\setcounter{jeolmsubproblem}{0}
\subproblem
Предположим, что для некоторого $n > 1$ существует разнообразная матрица~$A$
размера $n \times n$.
\par
Пусть число $x \in \{ 1, 2, \ldots, 2 n - 1 \}$ не~встречается среди элементов
главной диагонали $A$.
Такое число существует, так как на~диагонали не~более $n$ различных чисел.
Будем называть \emph{$i$-тым крестом} совокупность $i$-того столбца и~$i$-той
строки.
Всего крестов $n$, и~каждый из~них содержит ровно одно число~$x$.
С~другой стороны, каждый элемент матрицы~$A$, равный $x$, содержится ровно
в~двух крестах.
Следовательно, $n$ должно быть четным и, так как число $1997$ нечетно,
то~разнообразных матриц для $n = 1997$ не~существует.
\par
\subproblem
Для $n = 2$ и~$n = 4$ такие матрицы существуют:
\[
    A_2
=
    \begin{pmatrix}
        1 & 2 \\
        3 & 1
    \end{pmatrix}
\qquad
    A_4
=
    \begin{pmatrix}
        1 & 2 & 5 & 6 \\
        3 & 1 & 7 & 5 \\
        4 & 6 & 1 & 2 \\
        7 & 4 & 3 & 1
    \end{pmatrix}
\]
Эту конструкцию можно обобщить.
Предположим, что $n \times n$ матрица $A_n$~--- разнообразная.
Матрица $B_n$ получается из~$A_n$ добавлением $2 n$ к~каждому элементу, а~$C_n$
получается из~$B_n$ заменой всех диагональных элементов (по~индукции они равны
$2 n + 1$) на~$2 n$.
Тогда матрица
\[
    A_{2n}
=
    \begin{pmatrix}
        A_n & B_n \\
        C_n & A_n
    \end{pmatrix}
\]
разнообразная.
Докажем это.
Рассмотрим $i \leq n$ (случай $i > n$ аналогичен).
\par
В~нашей матрице $i$-й крест состоит из~$i$-го креста $A_n$, $i$-й
строки $B_n$ и~$i$-го столбца $C_n$.
Числа $1, 2, \ldots, 2 n - 1$ содержатся в~$i$-м кресте матрицы $A_i$.
В~$i$-й строке $B_n$ содержатся все числа вида $2 n + j$, где $j$ принадлежит
$i$-й строке $A_n$ (включая $j = 1$).
Аналогично, $i$-й столбец $C_n$ содержит $2n$ и~все числа вида $2 n + k$, где
$k > 1$ принадлежат $i$-му столбцу $A_n$.
Итак, мы~видим, что все числа $\{1, 2, \ldots, 4 n - 1\}$ содержатся в~$i$-м
кресте $A_{2n}$, и, значит, матрица $A_{2n}$ разнообразная.
Следовательно, разнообразные матрицы существуют для всех $n$, являющихся
степенями $2$.
\par
\emph{Второе решение для пункта b.}
Построим разнообразную матрицу для $n = 2^k$.
По~ходу решения будем рассматривать индексы по~модулю~$n$.
Определим $i$-диагональ ($0 \leq i < n$) как множество элементов матрицы
с~индексами $(j, j + i)$ для всех $j$.
Заметим, что каждый крест содержит ровно один элемент $0$-диагонали
(главной диагонали) и~по~два элемента на~остальных диагоналях.
Два элемента диагонали будем называть родственными, если некоторый крест
их~содержит.
Понятно, что у~всякого элемента, не~лежащего на~$0$-диагонали, ровно два
родственных.
Таким образом, отношение родственности разбивает каждую $i$-диагональ ($i > 0$)
на~циклы длины больше $1$.
Из-за симметрии по~модулю $n$ все циклы имеют равную длину, и, так как
$n = 2^k$, то~эта длина четна.
\par
Теперь очевидно, как строить разнообразную матрицу.
Выбираем число, скажем, $1$, и~берем все элементы $0$-диагонали равными этому
числу.
Оставшиеся числа разбиваем на~пары и~каждой диагонали сопоставляем свою пару
(скажем, $i$-диагонали сопоставляем пару $(2 i, 2 i + 1)$).
Проходя вдоль каждого цикла $i$-диагонали, по-очереди расставляем значения
$2 i$ и~$2 i + 1$.
Так как длины всех циклов четны, то~каждый элемент будет родственным только
отличным от~него элементам, а~значит, что каждый крест содержит как $2 i$, так
и~$2 i + 1$.
\fi % \ifincludesolutions

\begin{problems}

\item\textbf{(1998, UKR)}
Дана прямоугольная матрица.
Сумма чисел в~каждой строчке и~каждом столбце целая.
Докажите, что каждое нецелое число~$x$ в~матрице может быть заменено либо
на~$\lceil x \rceil$, либо на~$\lfloor x \rfloor$ так, что суммы чисел
в~строчках и~столбцах останутся неизменными.
(За~$\lceil x \rceil$ обозначим наименьшее целое число,
большее либо равное $x$,
а~за~$\lfloor x \rfloor$~--- наибольшее целое число,
меньшее либо равное $x$.)

\end{problems}

\ifincludesolutions
Очевидно, мы~можем заменить $x$ на~$\lfloor x \rfloor$ или $\lceil x \rceil$
так, что сумма чисел по~столбцам останется неизменной.
Тем не~менее, этого недостаточно, чтобы с~суммами по~строчкам было все так~же,
поэтому давайте рассмотрим сумму $S$ модулей изменений сумм в~строчках.
Очевидно, что $S$~--- неотрицательная, и~мы~хотим, чтобы она была 0.
\par
Сумма в~строке может увеличится или уменьшится при замене
(или остаться неизменной).
Пометим ячейку знаком $-$, если ее~содержимое $x$ было изменено
на~$\lfloor x \rfloor$, и~знаком $+$, если оно было изменено
на~$\lceil x \rceil$.
Назовем строчку~$R_2$ \emph{доступной} для строчки~$R_1$, если есть такой
столбец~$C$, что $C \cap R_1$ помечено знаком~$+$,
а~$C \cap R_2$ помечено знаком~$-$.
Заметим, что столбец, содержащий $+$, должен содержать также и~$-$, так как
сумма столбца не~изменяется.
Следовательно, для каждой строчки с~увеличившейся суммой мы~имеем доступ
к~какой-то другой строчке.
\par
Предположим, что сумма в~строчке~$R_1$ увеличилась.
Если $R_1, R_2, \ldots, R_k$~--- последовательность строчек, где $R_{i+1}$
доступна для $R_i$ через некоторый столбец~$C_i$ и~такая, что сумма
в~строчке~$R_k$ уменьшилась, тогда изменением знака в~$C_i \cap R_i$
и~$C_i \cap R_{i+1}$ ($i = 1, 2, \ldots, k - 1$) мы~уменьшаем $S$ на~2,
оставляя суммы в~столбцах неизменными.
Мы~утверждаем, что такая последовательность строчек всегда существует.
\par
Пусть $\mathcal{R}$~--- множество всех строчек, доступных для $R_1$, прямо
и~косвенно;
а~$\overline{\mathcal{R}}$~--- множество оставшихся строчек.
Покажем, что для любого столбца~$C$ сумма в~$\mathcal{R} \cap C$
не~увеличилась.
Если $\mathcal{R} \cap C$ не~содержит плюсы, тогда это очевидно.
Если $\mathcal{R} \cap C$ содержит $+$, то~так как строчки
$\overline{\mathcal{R}}$ не~доступны, множество $\overline{\mathcal{R}} \cap C$
не~содержит минусы.
Из~этого следует, что сумма в~$\overline{\mathcal{R}}\cap C$ не~увеличилась,
и~так как суммы в~столбце неизменны, мы~снова приходим к~тому~же выводу.
\par
Таким образом, общая сумма в~$\mathcal{R}$ не~увеличилась.
Следовательно, существует строчка из~$\mathcal{R}$ с~уменьшившейся суммой,
оправдывающей наше утверждение.
\fi % \ifincludesolutions

\begin{problems}

\item\textbf{(2004, POL)}
Дано четное натуральное число~$n$.
Рассмотрим множество матриц размера $n \times n$, элементы которых~---
вещественные числа, по~модулю не~превосходящие $1$, таких, что сумма элементов
в~каждой матрице равна $0$.
Найдите наименьшее число~$C$ такое, что в~каждой такой матрице найдется линия
(столбец или строка), сумма элементов в~которой по~модулю не~превосходит $C$.

\end{problems}

\ifincludesolutions
Рассмотрим матрицу $A = (a_{ij})_{i,j=1}^n$, элементы которой $a_{ij}$ равны
$1$ если $i, j \leq n / 2$, $-1$ если $i, j > n / 2$ и~0 в~остальных случаях.
Такая матрица удовлетворяет условиям задачи, суммы элементов в~ее~строках
и~столбцах равны $\pm n / 2$.
Следовательно $C \geq n / 2$.

Покажем, что $C = n / 2$.
Предположим, что это не~так, и~существует матрица $B = (b_{ij})_{i,j=1}^n$,
все суммы элементов в~строках и~столбцах которой либо больше $n / 2$, либо
меньше $- n / 2$.
Без ограничения общности можно считать, что по~крайней мере $n / 2$ сумм
по~строкам положительны и~что первые $n / 2$ строк имеют положительные суммы
(если это не~так, то~переставим строки).
Сумма элементов $n / 2 \times n$ подматрицы $B'$, состоящей из~первых $n / 2$
строк, превосходит $n^2 / 4$.
Так как сумма элементов в~каждом столбце матрицы $B'$ не~больше $n / 2$,
то~более половины столбцов матрицы~$B'$, а~значит и~$B$, имеют положительные
суммы.
Снова можно считать, что суммы первых $n/2$ столбцов положительны.
Итак, суммы $R^+$ и~$C^+$ элементов в~первых $n / 2$ строках и~первых $n / 2$
столбцах больше, чем $n^2 / 4$.
\par
Сумма всех элементов $B$ может быть записана в~виде
\[
    \sum a_{ij}
=
    {R^+} + {C^+}
    +
    \sum_{\substack{i > n / 2 \\ j > n / 2}}
        a_{ij}
    -
    \sum_{\substack{i \leq n / 2 \\ j \leq n / 2}}
        a_{ij}
>
    \frac{n^2}{2} - \frac{n^2}{4}  -\frac{n^2}{4}
=
    0
.\]
Противоречие.
Итак, $C = n / 2$.
\fi % \ifincludesolutions

\begin{problems}

\item\textbf{(2004, IRN)}
Рассмотрим матрицу~$A$ размера $n \times n$.
Обозначим за~$X_i$ множество элементов в~ее $i$-той строке, а~за~$Y_j$~---
множество элементов в~ее $j$-том столбце, $1 \leq i, j \leq n$.
Будем называть $A$ \emph{серебряной}, если множества
$X_1, \ldots, X_n, Y_1, \ldots, Y_n$ различны.
Найдите наименьшее натуральное~$n$ такое, что существует серебряная матрица
размера $2004 \times 2004$ с~элементами из~множества $\{ 1, 2, \ldots, n \}$.

\end{problems}

\ifincludesolutions
Так как $X_i, Y_i$, $i = 1, \ldots, 2004$~--- это $4008$ различных подмножеств
множества $S_n = \{ 1, 2, \dots, n \}$, то~$2^n \geq 4008$, то~есть
$n \geq 12$.
\par
Предположим, что $n = 12$.
Пусть
$\mathcal{X} = \{ X_1, \ldots, X_{2004} \}$,
$\mathcal{Y} = \{ Y_1, \ldots, Y_{2004} \}$,
$\mathcal{A} = \mathcal{X} \cup \mathcal{Y}$.
В~$\mathcal{A}$ не~содержится ровно $2^{12} - 4008 = 88$ подмножеств $S_n$.
\par
Так как каждая строка пересекает каждый столбец,
то~$X_i \cap Y_j \neq \emptyset$ для всех $i, j$.
Предположим, что $\abs{X_i}, \abs{Y_j} \leq 3$ для некоторых $i, j$.
Поскольку тогда $\abs{X_i \cup Y_j} \leq 5$, то~ни~одно из~по~крайней мере
$2^7 > 88$ подмножеств $S_n \setminus (X_i \cup Y_j)$ не~может содержаться
ни~в~$\mathcal{X}$, ни~в~$\mathcal{Y}$, что невозможно.
Следовательно, или в~$\mathcal{X}$, или в~$\mathcal{Y}$ все множества содержат
4 элемента или больше.
Предположим без ограничения общности, что
\(
    k = \abs{X_l} = \min_i \abs{X_i} \geq 4
\).
Тогда
\[
    n_k
=
    \binom{12 - k}{0} + \binom{12-k}{1} + \ldots + \binom{12-k}{k-1}
\]
подмножеств $S \setminus X_l$ cодержат меньше $k$~элементов и~ни~одно из~них
не~принадлежит ни~$\mathcal{X}$ (так как $\abs{X_l}$ наименьшее
в~$\mathcal{X}$), ни~$\mathcal{Y}$ (так как не~пересекаются с~$X_l$).
Поэтому должно выполняться $n_k \leq 88$.
Так как $n_4 = 93$ и~$n_5 = 99$, то~$k \geq 6$.
Но~тогда ни~одно из~$\binom{12}{0} + \ldots + \binom{12}{5} = 1586$
подмножеств $S_n$, содержащих меньше 6~элементов, не~попадает в~$\mathcal{X}$,
следовательно хотя~бы $1586 - 88 = 1498$ из~них должны попасть в~$\mathcal{Y}$.
$1498$ их~дополнений (которые содержат более 6 элементов) также не~могут
содержаться в~$\mathcal{X}$, то~есть уже $3084$ подмножеств $S_n$ не~содержатся
в~$\mathcal{X}$.
Противоречие.
\par
Построим серебряную матрицу для $n = 13$.
Пусть
\[
    A_1
=
    \begin{pmatrix}
        1 & 1 \\
        2 & 3
    \end{pmatrix}
\quad\text{и}\quad
    A_m
=
    \begin{pmatrix}
        A_{m-1} & A_{m-1} \\
        A_{m-1} & B_{m-1}
    \end{pmatrix}
\quad\text{для $m = 2, 3, \ldots$,}
\]
где $B_{m-1}$~--- матрица размера $2^{m-1} \times 2^{m-1}$, все элементы
которой равны $m + 2$.
Легко проверяется по~индукции, что все матрицы~$A_m$ серебряные.
Более того, всякая квадратная подматрица $A_m$ размера больше чем $2^{m-1}$,
полученная пересечением первых строк с~первыми столбцами, также серебряная.
Так как $2^{10} < 2004 < 2^{11}$, то~мы~получили серебряную матрицу размера
$2004$ cо значениями в~$S_{13}$.
\fi % \ifincludesolutions

\endgroup % \def\abs

\endgroup % \providecommand\ifincludesolutions

