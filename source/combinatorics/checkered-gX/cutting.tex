% $date: 2014-11-10
% $timetable:
%   gX:
%     2014-11-10:
%       2:

\section*{Клетчатая комбинаторика --- 1: разрезания}

% $authors:
% - Владимир Шарич

\begingroup
\providecommand\ifincludesolutions{\iffalse}

% $matter[with-solutions]:
% - preamble package: subcaption
% - verbatim: \begingroup\newcommand\ifincludesolutions{\iftrue}
% - .[-with-solutions]
% - verbatim: \endgroup

\emph{Задачи из short list'ов.}

\ifincludesolutions
\textbf{Версия с решениями.}
\emph{Решения являются слабообработанным переводом с~английского.
Beware!}
\fi

\begin{problems}

\item\textbf{(1999, CAN)}
\\
\sp
Пусть прямоугольник $5 \times n$ можно покрыть с~помощью $n$ пятиклеточных
фигур вида \raisebox{-0.2ex}{\jeolmfigure[height=2ex]{pentamino-P}}.
Докажите, что $n$ четно.
\\
\sp
Докажите, что существует более $2 \cdot 3^{k-1}$ способов покрытия
прямоугольника $5 \times 2 k$, $(k \geq 3)$, с~помощью $2k$ таких фигур.
(Симметричные покрытия считаются различными.)

\end{problems}

\ifincludesolutions
\setcounter{jeolmsubproblem}{0}%
\sp
Покрасим первый, третий и~пятый ряды в~красный, а~остальные квадратики в~белый.
Всего $n$ фигур и~$3n$ красных квадратов.
Поскольку каждая фигура покрывает не~более трех красных квадратов, она должна
покрывать ровно 3 красных квадратика.
Отсюда следует, что два белых квадратика, покрытых фигурой, лежат в~одном ряду,
иначе их~будет не~менее трех.
Следовательно, каждый белый ряд разбивается на~пары квадратов, принадлежащих
одному кусочку.
Отсюда следует, что количество клеток в~ряду, равное $n$, четно.
\par
\sp
За~$a_k$ обозначим количество различных покрытий прямоугольника $5 \times 2k$.
За~$b_k$ обозначим количество покрытий, не~разбивающихся на~меньшие разбиения
вертикальной линией (без пересечения кусочков).
Легко видеть, что $a_1 = b_1 = 2$, $b_2 = 2$, $a_2 = 6 = 2 \cdot 3$, $b_3 = 4$,
и~далее, по~индукции, $b_{3k} \geq 4$, $b_{3k+1} \geq 2$ и~$b_{3k+2} \geq 2$.
Также $a_k = b_k + \sum_{i=1}^{k-1} b_i a_{k-i}$.
При $k \geq 3$ по~индукции
\[
    a_k
>
    2 + \sum_{i=1}^{k-1} 2 a_{k-i}
\geq
    2\cdot3^{k-1}+2a_{k-1}
\geq
    2\cdot 3^k
\,.\]
\fi % \ifincludesolutions

\begin{problems}

\item\textbf{(2000, ITA)}
\emph{Блок} представляет собой трехступенчатую лестницу ширины 2, построенную
из~двенадцати одинаковых единичных кубиков.
Найдите все целые $n$, для которых с~помощью таких блоков можно построить куб
со~стороной~$n$.

\end{problems}

\ifincludesolutions
Так как объем каждого блока равен $12$, сторона любого такого куба должна
делиться на~$6$.
Предположим, что куб со~стороной $n = 6 k$ может быть построен
из~$n^3 / 12 = 18 k^3$ блоков.
Введем систему координат, где куб будет задан как
$[0; n] \times [0; n] \times [0; n]$,
а~каждый единичный кубик
$[2 p, 2 p + 1] \times[2 q, 2 q + 1] \times[2 r, 2 r + 1]$ окрашен
в~черный цвет.
Всего таких кубиков ровно $n^3 / 8 = 27 k^3$.
Каждый блок покрывает либо один, либо три черных кубика, что в~любом случае
будет нечетным числом.
Из~этого следует, что общее число черных кубиков должно быть четно, что
означает, что $k$~--- четно.
Следовательно, $12 \mid n$.
\par
С~другой стороны, два блока могут быть сложены в~параллелепипед
$2 \times 3 \times 4$.
Используя такие новые блоки, можно легко построить куб со~стороной~12, и,
следовательно, любой куб со~стороной, делящейся на~12.
\fi % \ifincludesolutions

\begin{problems}

\item\textbf{(2002, ARM)}%
\jeolmlabel{combinatorics/cutting-gX:2002-ARM:problem}
Дано нечетное $n \in \mathbb{N}$.
Единичные квадраты доски $n \times n$ покрашены в~шахматном порядке, причем
углы черные.
Назовем \emph{уголком} фигуру
\raisebox{-0.2ex}{\jeolmfigure[height=2ex]{tromino-L}}
из~трех единичных квадратов.
При каких $n$ можно покрыть все черные квадраты неперекрывающимися уголками?
В~случае, когда это возможно, какое минимальное количество уголков понадобится?

\end{problems}

\ifincludesolutions
Пусть $n = 2 k + 1$.
Рассмотрим черные квадраты на~нечетных линиях: их~в~целом $(k + 1)^2$ и~никакие
два их~них не~могут быть покрыты уголком.
Таким образом, нам необходимо как минимум $(k + 1)^2$ уголков, что покрывает
в~общем $3 (k + 1)^2$ квадратов.
$3 (k + 1)^2$ больше, чем $n^2$ для $n = 1, 3, 5$, поэтому будем считать, что
$n \geq 7$.
\begin{figure}[ht]\begin{center}
\strut\hfill
    \begin{subfigure}{3.5cm}
    \jeolmfigure[width=\linewidth]{2002-ARM-solution-7}
    \caption{база индукции}
    \jeolmlabel{combinatorics/cutting-gX:2002-ARM:solution:fig:7}
    \end{subfigure}
\hfill
    \begin{subfigure}{4.5cm}
    \jeolmfigure[width=\linewidth]{2002-ARM-solution-p2}
    \caption{переход индукции}
    \jeolmlabel{combinatorics/cutting-gX:2002-ARM:solution:fig:p2}
    \end{subfigure}
\hfill\strut\par
\caption{пример (оценка сверху)
к~решению задачи~\jeolmref{combinatorics/cutting-gX:2002-ARM:problem}.}%
\jeolmlabel{combinatorics/cutting-gX:2002-ARM:solution:fig}
\end{center}\end{figure}
\par
Случай $n = 7$ может быть покрыт, как показано
на~рис.~\jeolmref{combinatorics/cutting-gX:2002-ARM:solution:fig:7}.
Для $n > 7$ это тоже можно сделать, это следует по~индукции
из~рис.~\jeolmref{combinatorics/cutting-gX:2002-ARM:solution:fig:p2}.
\fi % \ifincludesolutions

\begin{problems}

\item\textbf{(2004, EST)}%
\jeolmlabel{combinatorics/cutting-gX:2004-EST:problem}
Найдите все пары натуральных чисел $(m, n)$, такие что прямоугольник
$m \times n$ может быть покрыт
\emph{крюками}
\raisebox{-0.5ex}{\jeolmfigure[height=2.3ex]{2004-EST-hook}},
состоящими из~шести единичных квадратов.
Крюки можно поворачивать и~симметрично отражать.
Прямоугольник должен быть покрыт без дыр и~перекрытий.
Ни~один крюк не~должен вылезать за~пределы прямоугольника.

\end{problems}

\ifincludesolutions
Предположим, что прямоугольник $m \times n$ может быть покрыт <<крюками>>.
Для каждого крюка~$H$ существует единственный крюк~$K$, покрывающий
<<внутренний>> квадрат~$H$.
$H$ также покрывает внутренний квадрат~$K$, поэтому множество крюков
распадается на~пары $\{ H, K \}$, каждая из~которых образует одну из~двух фигур
(будем называть их~\emph{плитками}) площади $12$, изображенных
на~рис.~\jeolmref{combinatorics/cutting-gX:2004-EST:solution:fig}.
\begin{figure}[ht]\begin{center}
\strut\hfill
    \begin{subfigure}{0.2\linewidth}\begin{center}
    \jeolmfigure[width=\linewidth]{2004-EST-solution-A1} $A_1$
    \end{center}\end{subfigure}
\hfill
    \begin{subfigure}{0.15\linewidth}\begin{center}
    \jeolmfigure[width=\linewidth]{2004-EST-solution-B1} $B_1$
    \end{center}\end{subfigure}
\hfill
    \begin{subfigure}{0.2\linewidth}\begin{center}
    \jeolmfigure[width=\linewidth]{2004-EST-solution-A2} $A_2$
    \end{center}\end{subfigure}
\hfill
    \begin{subfigure}{0.2\linewidth}\begin{center}
    \jeolmfigure[width=\linewidth]{2004-EST-solution-B2} $B_2$
    \end{center}\end{subfigure}
\hfill\strut
\caption{к~решению
    задачи~\jeolmref{combinatorics/cutting-gX:2004-EST:problem}}%
\jeolmlabel{combinatorics/cutting-gX:2004-EST:solution:fig}
\end{center}\end{figure}
Ясно, что наш прямоугольник должен покрываться такими плитками и~поэтому
$12 \mid m n$.
\par
Предположим, что одно из~чисел $m, n$ делится на~4.
Пусть, например, $4 \mid m$.
Если $3 \mid n$, то~прямоугольник легко покрывается прямоугольниками
$3 \times 4$, а~значит и~крюками.
Также, если $12 \mid m$ и~$n \not\in \{1, 2, 5\}$, то~существуют
$k,l\in\mathbb{N}_0$, такие что $n=3k+4l$, поэтому прямоугольник $m\times n$
можно разделить на~и~$3\times 12$ и~$4\times 12$, которые покрываются крюками.
Если~же $12\mid m$ и~$n=1,2$ или $5$, то~легко видеть, что покрытие крюками
невозможно.
\par Предположим теперь, что $4\nmid m$ и~$4\nmid n$.
Тогда $m,n$ четны, а~количество плиток нечетно.
Предположим, что общее количество плиток типов $A_1$ и~$A_2$ нечетно (иначе
общее количество плиток типов $B_1$ и~$B_2$ нечетно, что аналогично).
Покрасим в~черный цвет столбцы нашего прямоугольника, номера которых делятся на~4.
Тогда каждая плитка типа $A_1$ или $A_2$ покрывает три черных квадратика, и~всего этими плитками покрыто нечетное их~число.
Поэтому число черных квадратиков, покрытых плитками $B_1$ и~$B_2$, также должно
быть нечетным.
Но~это невозможно, так как каждая такая плитка покрывает либо два, либо четыре
черных квадратика.
\fi % \ifincludesolutions

\endgroup % \providecommand\ifincludesolutions

