% $date: 2014-11-17
% $timetable:
%   g10r2:
%     2014-11-17:
%       3:

\section*{Полуинварианты. Добавка}

% $authors:
% - Юлий Тихонов

% $matter[-contained,no-header]:
% - verbatim: \setproblem{8}
% - .[contained]

\begin{problems}

\item
Дано $n$ точек, никакие три из~которых не~лежат на~одной прямой.
Постройте несамопересекающуюся ломаную с~узлами в~этих точках.

\item
В строчку выписаны $n$ натуральных чисел.
Разрешается взять любые два числа $a$ и $b$ такие, что $a$ стоит левее $b$ и
$b$ не кратно $a$, и заменить $a$ на $(a, b)$, $b$ на $[a, b]$.
Докажите, что такие операции не могут проводиться бесконечно много раз.

\item
Есть куча из $n$ камней.
Разрешается заменять кучу на любое количество куч с меньшим количеством камней
(возможно, различным в разных кучах).
Докажите, что такие операции не могут проводиться бесконечно много раз.

\item
На~плоскости даны $n$~точек и~$n$~попарно непараллельных прямых.
Пронумеруйте точки и~прямые числами от~1 до~$n$ так, чтобы отрезки
перпендикуляров, опущенных из~соответствующих точек на~соответствующие прямые,
соединяющие точки с~одинаковыми номерами, не~пересекались.

\item
По одной стороне бесконечного коридора расположено бесконечное число комнат,
занумерованных по порядку целыми числами, и в каждой стоит по роялю.
В этих комнатах живет некоторое количество пианистов
(в одной комнате могут жить несколько пианистов).
Каждый день какие-то два пианиста, живущие в соседних комнатах~--- $k$-й и
$(k + 1)$-й, приходят к выводу, что они мешают друг другу и переселяются
соответственно в $(k - 1)$-ю и $(k + 2)$-ю комнаты.
Докажите, что через конечное число дней эти переселения прекратятся.

\end{problems}

