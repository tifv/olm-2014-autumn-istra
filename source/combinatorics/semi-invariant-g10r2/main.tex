% $date: 2014-11-14
% $timetable:
%   g10r2:
%     2014-11-14:
%       1:

\section*{Полуинварианты}

% $authors:
% - Юлий Тихонов

\begin{problems}

\item
Если на~доске написан квадратный трехчлен $x^2 + b x + c$, Петя может выбрать
произвольное $d$ и~написать новый трехчлен $x^2 + (b + 2 d) x + (c + b d)$.
Может~ли на~доске после нескольких таких операций появиться трехчлен
$x^2 - 2000 x + 1\,000\,000$, если изначальный трехчлен имел свободный член
$-1$?

\item
В~квадрате $10 \times 10$ покрашено $9$~клеток.
Каждый день закрашивают те~клетки, у~которых не~менее двух соседних по~стороне
уже покрашены.
Докажите, что полностью квадрат закрашен никогда не~будет.

\end{problems}

\subsection*{Остановите процесс}

\begin{problems}

\item
Дан граф~--- несколько городов, соединенных дорогами так, что из~каждого города
выходит нечетное число дорог.
Некоторые из~городов раскрашены в~красный цвет, а~некоторые~--- в~белый.
В~городе может произойти революция, если большинство его соседей раскрашено
не~в~тот~же цвет, что он~сам.
\\
\sp
Каждый день ровно в~одном из~городов происходит революция, и~он~меняет цвет
на~тот, в~который раскрашено большинство его соседей.
Докажите, что в~конце концов революции прекратятся.
\\
\sp
Каждый день революция одновременно происходит во~всех городах, в~которых она
может произойти, и~они меняют цвет на~тот, в~который было раскрашено
большинство их~соседей.
Докажите, что начиная с~определенного момента любой город либо остановится
на~некотором цвете, либо будет менять цвет каждый ход.
% Алексей Канель

\item
На~книжной полке каким-то образом расставлены тома полного собрания сочинений
Васи Пупкина.
Пьяный библиотекарь пытается расставить их~по~порядку.
\\
\sp
Для этого он~берет два тома (не~обязательно соседних), которые стоят
относительно друг друга не~по~порядку (то есть больший номер раньше меньшего),
и~переставляет их~местами.
\\
\sp
Для этого он~берет какой-то том, стоящий не~на~своем месте, сдвигает несколько
промежуточных томов, и~ставит этот том на~место.
\\
Докажите, что в~конце концов он~расставит книги по~порядку.
% Алексей Канель

\end{problems}

\subsection*{Запустите процесс}

\begin{problems}

\item
В~парламенте у~каждого парламентария есть не~более трех врагов.
Разделите парламент на~две палаты так, чтобы у~каждого парламентария было
не~более одного врага в~его палате.

\item
На~плоскости даны $n$ красных и~$n$ синих точек.
Пронумеруйте точки каждого цвета числами от~1 до~$n$ так, чтобы отрезки,
соединяющие точки с~одинаковыми номерами, не~пересекались.

\item
На~окружности расставлено несколько положительных чисел, каждое из~которых
не~больше 1.
Докажите, что можно разделить окружность на~три дуги так, что суммы чисел
на~соседних дугах будут отличаться не~больше, чем на~1.
(Если на~дуге нет чисел, то~сумма на~ней считается равной нулю.)

\item
В~парламенте некоторые депутаты залепили некоторым другим пощечины, причем
каждый депутат залепил не~более
\quad
\sp одной
\quad
\sp двух
\quad
пощечин
(пощечины не~рефлексивны: если $A$ ударил $B$, то, возможно, не~наоборот).
\setcounter{jeolmsubproblem}{0}
Докажите, что можно разбить парламент на
\quad
\sp три
\quad
\sp пять
\quad
палат так, чтобы в~каждой палате не~было дерущихся пар депутатов.
\\
\emph{Пункты этой задачи сдаются одновременно.}

\end{problems}

