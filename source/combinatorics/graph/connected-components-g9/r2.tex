% $date: 2014-11-11
% $timetable:
%   g9r2:
%     2014-11-11:
%       3:

\section*{Компоненты связности}

% $authors:
% - Иван Митрофанов

\begin{problems}

\item
В~графе $100$ вершин, степень каждой вершины равна $k$.
При каком наибольшем~$k$ этот граф может быть несвязным?

\item
Какое максимальное количество ребер может быть в~несвязном графе на~$100$
вершинах?

\item
В~стране $100$ городов, среди которых столица, из~которой выходит $33$ дороги,
и~город Удельный, из~которого выходит одна дорога.
Из~всех остальных городов выходит ровно по~двадцать дорог.
Обязательно~ли из~столицы можно добраться до~Удельного?

\item
Куб состоит из~$n^3$ единичных кубических ячеек.
Какое наименьшее количество перегородок между кубиками нужно удалить для того,
чтобы из~любой ячейки можно было добраться до~поверхности куба?
Выходить за~поверхность куба не~надо, то~есть для $n = 1$ и~$n = 2$ ответ $0$.

\item
Докажите, что связных графов на~$100$ вершинах больше, чем несвязных.

\item
Турист, приехавший в~Москву на~поезде, весь день бродил по~городу.
Поужинав в~кафе на~одной из~площадей, он~решил вернуться на~вокзал, и~при этом
идти только по~улицам, по~которым он~шел до~этого нечетное число раз.
Докажите, что он~всегда сможет это сделать.

\end{problems}

