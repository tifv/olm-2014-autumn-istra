% $date: 2014-11-11
% $timetable:
%   g9r1:
%     2014-11-11:
%       2:

\section*{Компоненты связности}

% $authors:
% - Иван Митрофанов

\begin{problems}

\item
В~графе $101$ вершина, степени всех вершин равны $k$.
При каком наибольшем~$k$ этот граф может быть несвязным?

\item
В~стране $100$ городов, среди которых столица, из~которой выходит $33$ дороги,
и~город Удельный, из~которого выходит одна дорога.
Из~всех остальных городов выходит ровно по~двадцать дорог.
Обязательно~ли из~столицы можно добраться до~Удельного?

\item
Куб состоит из~$n^3$ единичных кубических ячеек.
Какое наименьшее количество перегородок между кубиками нужно удалить для того,
чтобы из~любой ячейки можно было добраться до~поверхности куба?
Выходить за~поверхность куба не~надо, то~есть для $n = 1$ и~$n = 2$ ответ $0$.

\item
В~стране $50$ городов, любые два соединены дорогой с~односторонним движением.
Докажите, что можно выбрать одну из~дорог и~поменять на~ней направление
движения так, чтобы из~любого города можно было приехать в~любой другой.

\item
Турист, приехавший в~Москву на~поезде, весь день бродил по~городу.
Поужинав в~кафе на~одной из~площадей, он~решил вернуться на~вокзал, и~при этом
идти только по~улицам, по~которым он~шел до~этого нечетное число раз.
Докажите, что он~всегда сможет это сделать.

\item
Ребра связного графа раскрашены в~$N$ цветов, причем из~каждой вершины выходит
ровно по~одному ребру каждого цвета.
В~графе удалили $N - 1$ ребро, причем никакие два удаленных ребра не~имеют
одинаковый цвет.
Докажите, что граф остался связен.

\end{problems}

