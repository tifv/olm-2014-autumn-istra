% $date: 2014-11-10
% $timetable:
%   g10r1:
%     2014-11-10:
%       3:

% $caption: |-
%   Графы --- 1: n-связность (теорема Менгера)

\section*{Графы --- 1: $n$-связность (теорема Менгера)}

% $matter[g10r1,no-header,add-toc]:
% - verbatim: \phantomsection
% - verbatim: |-
%     \addcontentsline{toc}{section}{%
%       Графы --- 1: \texorpdfstring{$n$}{n}-связность (теорема Менгера)}
% - .[-add-toc]

% $authors:
% - Владимир Шарич

\definition
Пусть $s$ и~$t$~--- две различные вершины связного графа~$G$.
Две простых пути, соединяющих $s$ и~$t$, называются \emph{непересекающимися}
(или \emph{вершинно непересекающимися}), если у~них нет общих вершин, отличных
от~$s$ и~$t$ (и, следовательно, нет общих ребер),
и~\emph{реберно непересекающимися}, если у~них нет общих ребер.
Множество $W$ вершин, ребер или вершин и~ребер \emph{разделяет} $s$ и~$t$, если
$s$ и~$t$ принадлежат различным компонентам связности графа $(G - W)$.

\claim{Замечание}
Ясно, что не~бывает множеств вершин, разделяющих две смежные вершины.
Кроме того, очевидно, что вершин в~любом разделяющем $s$ и~$t$ наборе
не~меньше, чем путей в~любом наборе непересекающихся $s {-} t$~путей.

\subsection*{Теорема Менгера}

\theoremof{Менгера (1927 год)}
Наименьшее число вершин, разделяющих две несмежные вершины $s$ и~$t$, равно
наибольшему числу непересекающихся простых $s {-} t$~путей.

\emph{Доказательство теоремы Менгера (Дирак, 1966 год).}

\begin{problems}

\item
Докажите теорему Менгера для пар вершин $s$ и~$t$, которые разделяются одной
вершиной~$w$.

\end{problems}

Предположим, что теорема неверна.
Тогда среди всех графов~$G$ и~пар вершин $s, t \in G$ найдется тот, который
\emph{последовательно} удовлетворяет трем условиям минимальности:
\\
\textbf{(1)}
наименьшее число вершин, разделяющих $s$ и~$t$, минимально
(обозначим его~$h$);
\\
\textbf{(2)}
число вершин в~графе~$G$ наименьшее среди всех графов, для которых
теорема Менгера неверна и~которые удовлетворяют предыдущему пункту;
\\
\textbf{(3)}
множество ребер в~$G$ обладает тем свойством, что удаление любого несмежного
ни~с~$s$, ни~с~$t$ ребра ведет к~уменьшению минимального числа вершин,
разделяющих $s$ и~$t$.

Из~последнего условия минимальности следует, что для любого ребра~$x$ можно
определить множество вершин $S(x)$, содержащее не~более $(h - 1)$ элемента
и~разделяющего $s$ и~$t$ в~графе $(G - x)$.

\begin{problems}

\item
Докажите, что для произвольного ребра $x = u v$ в~графе~$G$ множество
$S(x) \cup \{u\}$, равно как и~множество $S(x) \cup \{v\}$, разделяет
$s$ и~$t$; в~частности, $x$ содержится в~некотором $s {-} t$~пути.

\item
Докажите, что в~графе~$G$ нет вершин~$w$, смежных и~с~$s$, и~с~$t$.

\end{problems}

Пусть $W$~--- произвольный набор $h$~вершин, разделяющих $s$ и~$t$.

\begin{problems}

\item
Докажите, что любой $s {-} t$~путь содержит какую-нибудь вершину из~$W$.

\end{problems}

Осмысленно рассмотреть для любого $s {-} t$~пути момент его входа
в~множество~$W$ и~момент его выхода оттуда.
Поэтому назовем пути от~$s$ до~$w \in W$, не~содержащие вершин из~$W$
(кроме $w$), \emph{$s {-} W$~путями},
и, аналогично, $w {-} t$~пути ($w \in W$), <<напрямую>> (т.~е. не~проходя
через другие вершины $W$) соединяющие $W$ с~$t$, назовем
\emph{$W \! {-} t$~путями.}

\begin{problems}

\item
Докажите, что никакой $s {-} W$~путь ни~с~каким $W \! {-} t$~путем не~может
иметь общих вершин, не~принадлежащих~$W$.

\item
Докажите, что любая вершина $w \in W$ содержится и~в~некотором $s {-} W$~пути,
и~в~некотором $W \! {-} t$~пути.

\item
Докажите, что либо все $s {-} W$~пути состоят из~одного ребра каждый, либо все
$W \! {-} t$~пути состоят из~одного ребра каждый;
иными словами, любое разделяющее $s$ и~$t$ множество смежно хотя~бы с~одной
из~вершин $s$ и~$t$.

\item
Завершите доказательство теоремы Менгера.
\\
\emph{Указание.}
Пусть $s u_1 u_2 \ldots t$~--- кратчайший $s {-} t$~путь в~$G$.
Положим $x = u_1 u_2$.
Что можно сказать о~$S(x) \cup \{u_1\}$ и~$S(x) \cup \{u_2\}$?

\end{problems}

\subsection*{Задачи}

\begin{problems}

\item
Докажите, что теорема Менгера равносильна такой теореме:
\begin{quote}
Для любых двух непересекающихся непустых множеств вершин $V_1$ и~$V_2$, таких
что никакая $v_1 \in V_1$ не~смежна ни~с~какой $v_2 \in V_2$, наибольшее число
непересекающихся путей, соединяющих $V_1$ и~$V_2$, равно наименьшему числу
вершин, разделяющих $V_1$ и~$V_2$.
\end{quote}
(Два простых пути, соединяющих $V_1$ и~$V_2$, называются
\emph{непересекающимися,} если они не~имеют общих вершин, отличных от~концевых
вершин.)

\item
Между зажимами $A$ и~$B$ включено несколько сопротивлений.
Каждое сопротивление имеет входной и~выходной зажимы.
Какое наименьшее число сопротивлений необходимо иметь и~какова может быть схема
их~соединения, чтобы при порче любых $9$ сопротивлений цепь оставалось
соединяющей зажимы $A$ и~$B$, но~не~было короткого замыкания?
(Порча сопротивления~--- это либо короткое замыкание, либо обрыв.)
\emph{(Мосгор 1958\,г.)}

\item
В~некотором городе для любых трех перекрестков $A$, $B$ и~$C$ есть путь,
ведущий из~$A$ в~$B$ и~не~проходящий через $C$.
Докажите, что с~любого перекрестка на~любой другой ведут по~крайней мере два
непересекающихся пути (перекресток~--- место, где сходятся по~крайней мере две
улицы; в~городе не~меньше двух перекрестков).
\emph{(Всерос 1966г.)}

\end{problems}

\definition
Связный граф~$G$ называется \emph{$n$--связным}, если\ldots
\\
\textbf{(1)}
\ldots он~остается связным при выкидывании любых $(n - 1)$ вершин.
\\
\textbf{(2)}
\ldots любые две вершины соединяются хотя~бы $n$~непересекающимися путями.

\begin{problems}

\item
Докажите эквивалентность двух определений $n$--связности.

\item\emph{Теорема Дирака.}
Граф, имеющий не~менее $2 n$~вершин, $n$--связен тогда и~только тогда, когда
для любых двух непересекающихся множеств $V_1$ и~$V_2$, в~каждом из~которых
по~$n$ вершин, существует $n$ соединяющих их~непересекающихся путей (имеющих,
в~частности, разные концы).

\item\emph{Теорема Форда--Фалкерсона (1955 год).}
Наименьшее число ребер, разделяющих две несмежные вершины $s$ и~$t$, равно
наибольшему числу реберно непересекающихся простых $s {-} t$ путей.

\end{problems}

