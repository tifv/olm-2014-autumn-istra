% $date: 2014-11-18
% $timetable:
%   g10r1:
%     2014-11-18:
%       3:

\section*{Графы --- 4: практикум (минимаксные теоремы)}

% $authors:
% - Владимир Шарич

\begin{problems}

\item
Докажите лемму Холла индукцией по количеству вершин.

\item
Даны натуральные числа $k \leq m < n$.
В графе~$G$ степени всех вершин не менее $m$ и не более $n$.
Докажите, что можно выкинуть несколько ребер, чтобы степени стали не менее
$m - k$ и не более $n - k$.

\item
В летний лагерь приехало некоторое количество школьников, причем каждый имеет
от 50 до 100 знакомых среди остальных.
Докажите, что вожатый Гриша сможет раздать им шапочки 1331 цветов так, чтобы
у каждого школьника среди его знакомых было не менее 20 различных цветов.

\item
Фокусник с помощником собираются показать такой фокус.
Зритель пишет на доске последовательность из $N$~цифр.
Помощник фокусника закрывает две соседних цифры черным кружком.
Затем входит фокусник.
Его задача --- отгадать обе закрытые цифры (и порядок, в котором они
расположены).
При каком наименьшем $N$ фокусник может договориться с помощником так, чтобы
фокус гарантированно удался?

\item
Ваня выписал на доску все натуральные делители числа 120, после чего стер
некоторые из них.
Оказалось, что среди оставшихся чисел нет двух таких, что одно делится
на другое.
Какое наибольшее число чисел могло остаться на доске?

\item
У Вовы есть набор из всевозможных карточек, на каждой из которых написано
несколько различных натуральных чисел от 1 до 100.
\\
\sp
Сколько всего карточек у Вовы?
\\
\sp
Вова разложил все свои карточки на несколько кучек так, что среди карточек
одной кучки нет двух таких, что все числа одной карточки встречаются на другой.
Какое наименьшее число кучек у него могло получиться?
\\
\sp
Какое наибольшее число карточек могло быть у Вовы в одной кучке?

\item
Дан 1001 прямоугольник с натуральными сторонами не больше 1000.
Докажите, что можно выбрать три из них $A$, $B$ и $C$ такие, что $A$ можно
поместить в $B$, а $B$ можно поместить в $C$.

\end{problems}

