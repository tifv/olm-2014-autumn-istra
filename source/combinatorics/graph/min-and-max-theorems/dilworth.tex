% $date: 2014-11-17
% $timetable:
%   g10r1:
%     2014-11-17:
%       1:

\section*{Графы --- 3: упорядоченные множества (теорема Дилуорса)}

% $authors:
% - Владимир Шарич

\begin{problems}

\item
Из~любых~ли
\quad
\sp девяти
\quad
\sp десяти
\quad
различных чисел, выписанных в~ряд, можно выбрать четыре, стоящие в~этом ряду
в~порядке убывания или в~порядке возрастания?
(Обобщение этой задачи на~$m n + 1$ чисел~---
выбираем $m + 1$ либо $n + 1$~--- назвается
теоремой \'{Э}рдёша~---~С\'{е}кереша.)

\item
На~прямой даны 50 отрезков.
Докажите, что если среди них нельзя найти 8 отрезков, каждые два из~которых
имеют общую точку, то~среди данных 50 отрезков можно найти 8 отрезков, никакие
два из~которых не~имеют общей точки.

\end{problems}

\definition
\emph{Частично упорядоченное} (или просто \emph{упорядоченное})
множество~$M$~--- это множество, для любых двух элементов $a$ и~$b$ которого
известно, находятся они в~некотором отношении $\prec$ или нет.
При этом должны быть выполнены следующие свойства:
\\
\textbf{(1)}
$c \prec c$ \emph{(рефлексивность)};
\\
\textbf{(2)}
если $a \prec b$ и~$b \prec a$, то~обязательно $a = b$
\emph{(антисимметричность)};
\\
\textbf{(3)}
если $a \prec b$ и~$b \prec c$, то~$a \prec c$ \emph{(транзитивность)}.

\begin{problems}

\item
Докажите, что в~частично упорядоченном множестве из~$m n + 1$ элементов есть
либо \emph{цепь} (множество, все элементы которого сравнимы) из~идущих
в~порядке возрастания или убывания $m + 1$ элементов, либо $n + 1$ попарно
несравнимых элементов \emph{(антицепь).}

\item\textbf{Теорема Мирского, 1971г.}
Обозначим через $d$ наибольшее количество элементов цепи данного конечного
частично упорядоченного множества~$M$.
Тогда $M$ можно разбить на~$d$~антицепей.

\item
Король сказочной страны пригласил на~пир всех людоедов своей страны.
Среди них есть людоеды, которые хотят съесть других людоедов.
Известно, что наидлиннейшая цепочка, в~которой первый людоед хочет съесть
второго, второй~--- третьего и~так далее, состоит из~$n$ людоедов.
Докажите, что король может так рассадить людоедов за~$n$ столов, что ни~за~каким столом никто не~будет хотеть съесть никого из~сидящих за~тем~же столом.

\item\textbf{%
Теорема Дилуорса,\footnote{Robert P. Dilworth}
1950г.}
Наименьшее количество цепей, которые покрывают данное конечное частично
упорядоченное множество~$M$, равно наибольшему количеству попарно несравнимых
элементов (т.~е. наибольшей антицепи).
\par
\emph{Воспользуйтесь теоремой Кенига для двудольного графа, обе доли которого
совпадают c $M$, а~ребра\ldots{} догадайтесь :)}

\item
Даны несколько различных натуральных чисел.
Пусть среди любых $n$ из~них можно выбрать два так, что одно делится на~другое.
Докажите, что все числа можно покрасить в~$(n - 1)$~цветов так, чтобы из~любых
двух чисел одного цвета одно делилось на~другое.

\end{problems}

