% $date: 2014-11-12
% $timetable:
%   g10r1:
%     2014-11-12:
%       2:

\section*{Графы --- 2: паросочетания (теоремы Кенига, Холла)}

% $authors:
% - Владимир Шарич

\begingroup
    \def\abs#1{\lvert #1 \rvert}

Рассматриваем двудольный граф $G = V \cup U$, где $V$ и~$U$~--- доли.
Часто $V$ будем называть \emph{первой,} а~$U$~--- \emph{второй долей.}
Множество всех ребер графа $G$ обозначим традиционно $E$.

\definition
\emph{Паросочетанием} называется набор ребер~$P$, в~котором никакие два ребра
не~имеют общих вершин.
При этом мы~будем писать, что вершина $v \in P$, если она смежна хотя~бы одному
ребру из~$P$.
\emph{Вершинным покрытием} называется такой набор вершин, что любое ребро
смежно хотя~бы одной вершине из~этого набора.

\begin{problems}

\item\textbf{Теорема Кенига.\footnote{Dénes Kőnig}}
Если $\alpha_1$~--- количество ребер в~наибольшем паросочетании,
а~$\beta_0$~--- количество вершин в~наименьшем вершинном покрытии,
то~$\alpha_1 = \beta_0$.

\item\emph{Теорема Кенига.}
В~обществе из~женихов и~невест нельзя образовать $n + 1$ брак по~любви.
Докажите, что можно выбрать $n$ человек так, чтобы из~любой пары влюбленных был
кто-то выбран.

\item\emph{Теорема Кенига-Эгервари.}
На~бесконечной клетчатой доске стоят ладьи.
\emph{Линией} называется строка или столбец.
Оказалось, что $N$~--- минимальное число линий, содержащих все ладьи.
Докажите, что можно выбрать $N$ ладей, не~бьющих друг друга.

\item\textbf{Теорема Холла.\footnote{Philip Hall}}
Для любого $A \subseteq V$ определим $F(A)$ как множество всех вершин из~второй
доли, которые соединены хотя~бы с~одной вершиной из~$A$:
\[
    F(A)
=
    \{
        u \in U
    \colon
        \exists v \in A
    ,\,
        vu \in E
    \}
\,.\]
\subproblem
Покажите, что если существует паросочетание, покрывающее все вершины доли~$V$,
то~выполнено условие
\[
    \forall A \subseteq V
\quad
    \abs{F(A)} \geq \abs{A}
\,.\]
\subproblem
Докажите, что если условие из~предыдущего пункта выполнено, то~тогда найдется
паросочетание, полностью покрывающее $V$.

\end{problems}

\claim{Замечание}
Доли не~равноправны, теорема не~имеет <<симметричный>> характер!

\begin{problems}

\item\emph{Теорема Холла.}
Пусть имеется набор конечных множеств $S_1, S_2, \ldots, S_n$.
\emph{Системой различных представителей} называется набор
$x_1, x_2, \ldots, x_n$, $x_i \in S_i$ такой, что
\(
    i \neq j
\;\Rightarrow\;
    x_i \neq x_j
\).
Система различных представителей существует тогда и~только тогда, когда
\[
    \forall k \geq 1
\;
    \forall i_1, \ldots, i_k
\quad
    \abs{S_{i_1} \cup \ldots \cup S_{i_k}} \geq k
\;.\]

\end{problems}

\definition
\emph{Латинским прямоугольником} называется числовой прямоугольник (таблица)
$n \times m$, $n \geq m$, в~который записаны числа $1, 2, \ldots, n$, причем
в~любой строчке и~любом столбце нет равных чисел.

\begin{problems}

\item
Докажите, что латинский прямоугольник всегда можно дополнить до~латинского
квадрата.

\item
В~кубе $8 \times 8 \times 8$ несколько нижних слоев заполнены не~бьющими друг
друга ладьями.
(Заполнены~--- это значит, что в~каждом слое находится максимальное число
ладьей~--- 8.)
Докажите, что можно заполнить оставшиеся слои с~сохранением того свойства, что
ладьи не~бьют друг друга.

\item
Труппа из~16 человек играет пьесу, в~которой 16 ролей.
Один актер в~каждом спектакле играет ровно одну роль, и~роли у~каждого актера
не~повторяются в~одном сезоне.
Труппа заканчивает сезон, когда нельзя подобрать роли актерам, удовлетворяющие
этому требованию.
Может~ли труппа закончить сезон быстрее, чем за~16 спектаклей?
\emph{(ФЮМ, 1995г.)}

\item
Квадратный лист бумаги разбит на~сто многоугольников одинаковой площади с~одной
стороны и~на~сто других той~же площади с~обратной стороны.
Докажите, что этот квадрат можно проткнуть ста иголками так, что каждый
из~двухсот многоугольников проткнут по~разу.

\item
Докажите, что в~регулярном двудольном графе есть паросочетание, покрывающее
каждую вершину ровно один раз \emph{(1-фактор)}.

\item
Докажите, что регулярный двудольный граф разбивается на~1-факторы.

\item\emph{Теорема Петерсена.}
Дан регулярный граф четной степени.
\\
\subproblem
Докажите, что в~таком графе есть цикл, проходящий по~всем ребрам ровно
по~одному разу.
Далее ориентируем граф так, чтобы этот цикл был ориентированным.
\\
\subproblem
<<Раздвоим>> каждую вершину на~<<вершину для входа ребер>>
и~<<вершину для выхода ребер>>.
Докажите, что в~получившемся двудольном графе есть 1-фактор.
\\
\subproblem
Докажите, что все вершины можно покрыть несколькими непересекающимися циклами
(т.~е. можно выделить \emph{2-фактор}).

\item
Есть юноши и~девушки.
Каждый юноша знаком с~некоторыми девушками.
Рассмотрим какое-нибудь множество из~$k$~юношей.
Пусть количество девушек которые знают хотя~бы одного из~этих юношей~---
$k_1$, а~количество девушек которые знают хотя~бы двух из~них~--- $k_2$.
Назовем это множество \emph{хорошим}, если $k_1 + k_2 \geq 2 k$.
Докажите, что если любое множество юношей хорошее, то~каждому юноше можно найти
знакомую жену и~знакомую любовницу (они должны быть разными), так что каждая
девушка не~более чем для одного~--- жена и~не~более чем для одного~---
любовница.

\item
Дано множество мощности $(2 k + 1)$.
Пусть $X$~--- набор всех его $k$--элементных подмножеств.
Докажите, что есть биекция $f \colon X \to X$, такая, что
\[
    \forall y \in X
\quad
    y \cap f(y) = \varnothing
\,.\]

\item
Есть $n$ юношей и~$2 n - 1$ девушка.
Каждому юноше нравятся некоторые девушки.
Докажите, что каждому юноше можно найти жену так, что или она ему нравится, или
ему не~нравится ни~одна из~чужих жен.

\item
Докажите, что из~52 натуральных чисел, не~превосходящих 100, можно выбрать
6~чисел, любая пара из~которых отличается в~обоих разрядах.

\end{problems}

\endgroup % \def\abs

