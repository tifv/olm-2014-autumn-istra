% $date: 2014-11-13
% $timetable:
%   g9r1:
%     2014-11-13:
%       2:

\section*{Планарные графы}

% $authors:
% - Иван Митрофанов

\begin{problems}

\item
Каждая грань выпуклого многогранника является либо пятиугольником, либо
шестиугольником.
Каким может быть количество пятиугольников?

\item
Докажите, что в~любом планарном графе найдется вершина степени~$5$ или меньше.

\item
Ребра полного графа на~$11$~вершинах покрасили в~два цвета.
Может~ли быть так, что ребра каждого цвета образуют планарный граф?

\item
Можно~ли ребра полного графа на~$8$ вершинах покрасить в~два цвета так, чтобы
ребра каждого цвета образовывали планарный граф?

\item
В~планарном графе все вершины имеют степень $4$.
По~ребрам графа ползет муравей, при этом он~на~каждом перекрестке поворачивает
либо влево, либо вправо.
Через некоторое время муравей прополз по~тому ребру, по~которому двигался
ранее.
Докажите, что он~полз в~том~же направлении.

\item
На~плоскости нарисовано несколько непересекающихся кругов, соединенных
непересекающимися отрезками;
из~каждого круга выходит четное число отрезков, и~вся конструкция связна.
Докажите, что существует замкнутый несамопересекающийся маршрут, проходящий
по~каждому отрезку и~по~некоторым дугам окружностей.

\item
Из~гипотезы четырех красок выведите следующее утверждение: если в~планарном
графе нет мостов и~степень каждой вершины равна $3$, то~его ребра можно
правильным образом раскрасить в~$3$ цвета.

\end{problems}

