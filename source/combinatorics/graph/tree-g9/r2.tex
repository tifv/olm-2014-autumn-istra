% $date: 2014-11-14
% $timetable:
%   g9r2:
%     2014-11-14:
%       2:

\section*{Деревья}

% $authors:
% - Иван Митрофанов

Все графы далее предполагаются без петель и~кратных ребер.

\begin{problems}

\item
Рыболовная сеть имеет форму квадрата $10 \times 10$.
Какое наибольшее количество веревочек можно перерезать так, чтобы сетка
не~распалась?

\item
Двое играют в~следующую игру: изначально есть полный граф на~$55$ вершинах,
игроки по~очереди удаляют по~одному ребру.
Проигрывает тот, кто получает несвязный граф.
Кто выигрывает при правильной игре?

\item
Может~ли в~графе быть ровно два остовных дерева?

\end{problems}

\definition
\emph{Точка сочленения}~--- вершина, при удалении которой вместе со~всеми
исходящими ребрами граф теряет связность.

\begin{problems}

\item
В~связном графе есть хотя~бы две вершины.
Докажите, что в~нем есть хотя~бы две вершины, не~являющиеся точками сочленения.

\item
В~связном графе $n$~вершин, степень каждой вершины равна~$3$.
Раскраска ребер графа в~$3$~цвета называется \emph{правильной}, если из~любой
вершины выходят ребра трех разных цветов.
Докажите, что правильных раскрасок не~больше, чем $3 \cdot 2^n$.

\end{problems}

\subsection*{Ещё деревья}

\begin{problems}

\item
В~дереве $101$ вершина.
Докажите, что можно удалить одну из~них вместе со~всеми выходящими ребрами так,
чтобы в~каждой из~оставшихся компонент связности было меньше $50$~вершин.

\item
В~стране $15$~городов, некоторые из~них соединены авиалиниями.
Каждая авиалиния принадлежит одной из~трех авиакомпаний, при этом при закрытии
любой из~трех авиакомпаний из~любого города в~любой можно будет добраться двумя
оставшимися компаниями.
Какое может быть наименшьшее количество авиалиний в~стране?

\item
В~дереве три вершины покрашены в~красный так, что для любой вершины~$x$ все три
красные вершины находятся на~расстоянии не~более, чем $10$.
Докажите, что существует такая вершина $A$, что все вершины дерева находятся
от~нее на~расстоянии не~превосходящем $9$.

\end{problems}

