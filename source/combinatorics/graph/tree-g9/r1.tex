% $date: 2014-11-12
% $timetable:
%   g9r1:
%     2014-11-12:
%       1:

\section*{Деревья}

% $authors:
% - Иван Митрофанов

\definition
\emph{Точка сочленения}~--- вершина, при удалении которой вместе со~всеми
исходящими ребрами граф теряет связность.

\definition
\emph{Мост}~--- ребро, при удалении которого граф теряет связность.

Все графы далее предполагаются без петель и~кратных ребер.

\begin{problems}

\item
Может~ли в~графе быть ровно два остовных дерева?

\item
В~связном графе есть хотя~бы две вершины.
Докажите, что в~нем есть хотя~бы две вершины, не~являющиеся точками сочленения.

\item
В~графе $n$~вершин, степень каждой вершины равна~$3$.
Раскраска ребер графа в~$3$~цвета называется \emph{правильной}, если из~любой
вершины выходят ребра трех разных цветов.
Докажите, что правильных раскрасок не~больше, чем $3 \cdot 2^n$.

\item
В~дереве три вершины покрашены в~красный так, что для любой вершины $x$ все три
красные вершины находятся на~расстоянии не~более, чем $10$.
Докажите, что существует такая вершина $A$, что все вершины дерева находятся
от~нее на~расстоянии не~превосходящем $9$.

\item
Есть дерево на~$n$ вершинах.
На~плоскости отмечены $n$ точек общего положения.
Докажите, что можно провести $n-1$ отрезок между этими точками так, чтобы
получившийся граф был изоморфен исходному дереву.

\item
У~дерева $100$ висячих вершин.
Докажите, что можно провести $50$ новых ребер так, чтобы в~получившемся графе
не~было мостов.

\end{problems}

