% $date: 2014-11-17

% $timetable:
%   g10r2:
%     2014-11-17:
%       2:

\section*{Двудольные графы и теорема Холла}

% $authors:
% - Владимир Шарич

\definition
Если вершины графа можно раскрасить в~два цвета так, чтобы никакое ребро
не~соединяло вершины одинакового цвета, такой граф называется
\emph{двудольным}.

\begin{problems}

\item
Докажите, что граф двудольный тогда и~только тогда, когда он~не~содержит циклов
нечетной длины.

\item
Какое наибольшее количество ребер может быть в~двудольном графе
на~$n$~вершинах?

\item
На~вечере ни~один мальчик не~танцевал со~всеми девочками, а~каждая девочка
танцевала хотя~бы с~одним мальчиком.
Докажите, что существует 2 мальчика и~2 девочки такие, что первый танцевал
с~первой, второй~--- со~второй, а~первый со~второй и~второй с~первой
не~танцевали.

\item
Вершины конечного графа как-то пронумеровали от~1 до~$n$, затем на~каждом ребре
записали сумму номеров его концов, а~номера в~вершинах стерли.
Докажите, что если граф не~двудольный, то~нумерация однозначно
восстанавливается.

\item
Леша записал в~клетки шахматной доски числа $1, 2, \ldots, 64$ в~неизвестном
порядке.
Он~сообщил Юлию сумму чисел в~каждом прямоугольнике из~двух клеток и~добавил,
что $1$ и~$64$ лежат на~одной диагонали.
Докажите, что по~этой информации Юлий может точно определить, где какое число
записано.

\item
Пусть вершины графа~--- это узлы клетчатой бумаги, ребра~--- отрезки
фиксированной длины~$L$.
Докажите, что получившийся граф двудольный.

\item
Предположим, что в~графе~$G$ через каждую вершину проходит не~более $n$ простых
нечетных циклов.
Тогда граф можно правильно раскрасить в~$n + 2$ цветов.

\end{problems}

\subsection*{Теорема Холла}

Рассматриваем двудольный граф $G = V \cup U$, где $V$ и~$U$~--- доли.
Часто $V$ будем называть \emph{первой,} а~$U$~--- \emph{второй долей.}
Множество всех ребер графа $G$ обозначим традиционно $E$.

\definition
\emph{Паросочетанием} называется набор ребер~$P$, в~котором никакие два ребра
не~имеют общих вершин.
При этом мы~будем писать, что вершина $v \in P$, если она смежна хотя~бы одному
ребру из~$P$.

\begin{problems}

\item\textbf{Теорема Холла.\footnote{Philip Hall}}
Для любого $A \subseteq V$ определим $F(A)$ как множество всех вершин из~второй
доли, которые соединены хотя~бы с~одной вершиной из~$A$:
\[
    F(A)
=
    \{
        u \in U
    \colon
        \exists v \in A
    ,\,
        vu \in E
    \}
\,.\]
\subproblem
Покажите, что если существует паросочетание, покрывающее все вершины доли~$V$,
то~выполнено условие
\[
    \forall A \subseteq V
\quad
    |F(A)| \geq |A|
\,.\]
\subproblem
Докажите, что если условие из~предыдущего пункта выполнено, то~тогда найдется
паросочетание, полностью покрывающее $V$.

\end{problems}

\observation
Доли не~равноправны, теорема не~имеет <<симметричный>> характер!

\begin{problems}

\item\emph{Теорема Холла.}
Пусть имеется набор конечных множеств $S_1, S_2, \ldots, S_n$.
\emph{Системой различных представителей} называется набор
$x_1, x_2, \ldots, x_n$, $x_i \in S_i$ такой, что
\(
    i \neq j
\;\Rightarrow\;
    x_i \neq x_j
\).
Система различных представителей существует тогда и~только тогда, когда
\[
    \forall k \geq 1
\;
    \forall i_1, \ldots, i_k
\quad
    |S_{i_1} \cup \ldots \cup S_{i_k}| \geq k
\;.\]

\item
Докажите, что в~регулярном двудольном графе есть паросочетание, покрывающее
каждую вершину ровно один раз \emph{(1-фактор)}.

\end{problems}

\definition
\emph{Латинским прямоугольником} называется числовой прямоугольник (таблица)
$n \times m$, $n \geq m$, в~который записаны числа $1, 2, \ldots, n$, причем
в~любой строчке и~любом столбце нет равных чисел.

\begin{problems}

\item
Докажите, что латинский прямоугольник всегда можно дополнить до~латинского
квадрата.

\item
В~кубе $8 \times 8 \times 8$ несколько нижних слоев заполнены не~бьющими друг
друга ладьями.
(Заполнены~--- это значит, что в~каждом слое находится максимальное число
ладьей~--- 8.)
Докажите, что можно заполнить оставшиеся слои с~сохранением того свойства, что
ладьи не~бьют друг друга.

\item
Труппа из~16 человек играет пьесу, в~которой 16 ролей.
Один актер в~каждом спектакле играет ровно одну роль, и~роли у~каждого актера
не~повторяются в~одном сезоне.
Труппа заканчивает сезон, когда нельзя подобрать роли актерам, удовлетворяющие
этому требованию.
Может~ли труппа закончить сезон быстрее, чем за~16 спектаклей?
\emph{(ФЮМ, 1995г.)}

\item
Квадратный лист бумаги разбит на~сто многоугольников одинаковой площади с~одной
стороны и~на~сто других той~же площади с~обратной стороны.
Докажите, что этот квадрат можно проткнуть ста иголками так, что каждый
из~двухсот многоугольников проткнут по~разу.

\item
Докажите, что регулярный двудольный граф разбивается на~1-факторы.

\end{problems}

