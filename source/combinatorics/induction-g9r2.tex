% $date: 2014-11-13
% $timetable:
%   g9r2: {}

% $groups$matter$into: false
% $groups$delegate$into: false

\section*{Индукция}

% $authors:
% - Владимир Брагин

\begin{problems}

\itemy{0}
В~квадрате $2^n \times 2^n$ вырезана
\quad
\sp угловая
\quad
\sp произвольная
\quad
клетка.
Докажите, что оставшуюся доску можно разрезать на~уголки из~трех клеток.

\item
Докажите, что любую сумму, большую 7~рублей, можно выдать купюрами 3 и~5
рублей.

\item
Докажите, что для любого натурального $n \geq 6$ квадрат можно разрезать
на~$n$~квадратов.

\item
Несколько прямых разбивают плоскость на~части.
Докажите, что можно раскрасить эти части в~два цвета так, чтобы цвета
граничащих граней были различны.

\item
Вершины выпуклого многоугольника раскрашены в~три цвета так, что каждый цвет
присутствует и~никакие две соседние вершины не~окрашены в~один цвет.
Докажите, что многоугольник можно разбить диагоналями на~треугольники так,
чтобы у~каждого треугольника вершины были трех разных цветов.

\item
На~кольцевом шоссе стоят несколько автомобилей с~общим запасом бензина,
достаточным, чтобы объехать весь круг.
Докажите, что можно сесть в~один из~автомобилей и~проехать все шоссе, забирая
по~дороге бензин у~остальных автомобилей.

\item
В~стране $n$~городов, любые два из~них соединены односторонней дорогой.
Оказалось, что из~каждого города можно доехать до~каждого.
Докажите, что можно выехать из~любого города, посетить все города ровно
по~одному разу и~вернуться обратно.

\item
В~соревновании участвуют 32~боксера.
Каждый боксер в~течение одного дня может проводить только один бой.
Известно, что все боксеры имеют разную силу, и~что сильнейший всегда
выигрывает.
Докажите, что за~15 дней можно определить место каждого боксера.
(Расписание каждого дня соревнований составляется вечером накануне и~в~день
соревнований не~изменяется.)

\end{problems}

