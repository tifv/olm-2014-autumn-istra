% $date: 2014-11-15
% $timetable:
%   g9r2:
%     2014-11-15:
%       1:

\section*{Игры}

% $authors:
% - Иван Митрофанов

\begin{problems}

\item
Есть клетчатый прямоугольник $m \times n$.
Двое по~очереди закрашивают любую строку или любой столбец в~нем.
Закрашивать строку, где все клетки уже закрашены, нельзя.
Проигрывает тот, у~кого нет хода.
Кто выигрывает при правильной игре?

\item
На~плоскости отмечены $n$~точек, никакие три не~лежат на~одной прямой.
Двое соединяют стрелочками эти точки.
Стрелочку можно проводить из~конца предыдущей стрелочки в~любую отмеченную
точку, с~которой стрелочки еще не~было (ни~в~этом направлении, ни~в~обратном).
Проигрывает тот, кому некуда ходить.
Кто?

\item
На~доске $n \times n$ двое по~очереди ходят фишкой.
Сначала она стояла в~углу, ходить можно на~соседнюю по~стороне клетку,
на~которой фишка еще не~была.
Докажите, что при четном $n$ выигрывает первый, а~при нечетном~--- второй
игрок.

\item
Есть куб.
Первый красит три его ребра в~красный цвет, потом второй красит еще три ребра
в~синий цвет, потом первый красит три ребра в~красный, потом второй~---
оставшиеся в~синий.
Каждое ребро красить можно только один раз.
Выигрывает тот, кому удалось покрасить в~свой цвет ребра одной грани.
Кто выигрывает?

\item
Есть $17$ куч из~монет, в~первой одна, во~второй две и~т.~д.
Играют двое.
Ход такой: человек, у~которого монет больше (если поровну, то~ходивший
в~прошлый раз) выбирает еще не~выбранную кучу (в~первый раз ходящий избирается
жребием).
Его соперник решает, кому она достанется из~них двоих.
Далее следующий ход.
Выигрывает тот, у~кого в~конце больше монет.
Кто?

\item
На~доске написано число $2$.
Ход состоит в~увеличении числа хотя~бы на~один, но~не~более, чем в~два раза,
и~записывании его на~доску вместо старого.
Выигрывает тот, кто получил число $2014$.
Кто?

\end{problems}

