% $date: 2014-11-17
% $timetable:
%   g9r1:
%     2014-11-17:
%       2:

\section*{Комбинаторная теория чисел}

% $authors:
% - Владимир Брагин

\begin{problems}

\item
Какое наибольшее количество чисел можно выбрать среди первых $2014$ натуральных
чисел, чтобы сумма никаких двух чисел не~делилась на~их~разность?

\item
Какое наибольшее количество чисел можно выбрать среди первых $2014$ натуральных
чисел, чтобы никакое из~выбранных чисел не~делилась на~другое выбранное?

\item
На~какое наименьшее количество групп можно разбить числа от~1 до~2014 так,
чтобы среди чисел одной группы ни~одно из~чисел не~делилось ни~на~какое другое?

\item
Каждое натуральное число покрашено в~красный или синий цвет.
Оказалось, что произведение любых двух разноцветных чисел красное, а~сумма
синяя.
Какого цвета может быть произведение двух красных чисел?

\item
\sp
Докажите, что из~$n$ натуральных чисел можно выбрать несколько так, что
их~сумма будет делиться на~$n$.
\\
\sp
Вася выбрал $(n - 1)$ целых чисел, так что сумма любых несколькиx чисел
из~этого набора не~делится на~$n$.
Докажите, что все выбранные им~числа попарно сравнимы по~модулю $n$.
\\
\sp
Петя выбрал 100 натуральных чисел, каждое из~которых меньше 100 так, что сумма
всех чисел равна 200.
Докажите, что сумма нескольких из~них равна 100.

\item
\sp
Дано простое число~$p$.
Выбрали $(2 k - 1)$ чисел ($k \leq p$), среди которых нет $k + 1$ одинаковых,
и~посчитали остаток суммы любых $k$ из~них при делении на~$p$.
Докажите, что получилось не~меньше $k$ различных остатков.
\\
\sp
Докажите, что из~$(2 p - 1)$ целых чисел можно выбрать $p$, сумма которых
делится на~$p$.
($p$~--- простое).
\\
\sp\emph{Теорема Эрдеша-Гинзбурга-Зива.}
Докажите, что из~$(2 n - 1)$ целых чисел всегда можно выбрать $n$~чисел
c~суммой, кратной~$n$.

\item
Имеется много карточек, на~каждой из~которых записано натуральное число
от~1 до~$n$.
Известно, что сумма чисел на~всех карточках делится на~$n!$.
Докажите, что карточки можно разложить на~несколько групп так, чтобы в~каждой
группе сумма чисел, записанных на~карточках, равнялась $n!$.

\end{problems}

