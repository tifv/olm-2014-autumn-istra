% $date: 2014-11-12
% $timetable:
%   g9r2: {}

% $groups$matter$into: false
% $groups$delegate$into: false

% $matter[-contained,no-header]:
% - verbatim: \section*{Полуинварианты}
% - verbatim: \setproblem{6}
% - .[contained]

% $authors:
% - Владимир Брагин

\emph{Если из~любого состояния есть выход, а~состояний конечное число,
то~повтора не~избежать\ldots}

\begin{problems}

\item
На~плоскости отмечено~$n$ красных и~$n$ синих точек общего положения.
Докажите, что можно провести~$n$ непересекающихся отрезков,
каждый из~которых соединяет красную точку с~синей.

\item
Вдоль бесконечного в~обе стороны коридора, расположены комнаты, занумерованные
целыми числами.
В~этих комнатах живут пианисты (их~конечное число).
Каждый день какие-то два пианиста, живущие в~соседних комнатах ($n$ и~$n+1$)
решают, что мешают друг другу и~разъезжаются, то~есть переселяются в~комнаты
$n-1$ и~$n+2$.
(При этом несколько пианистов могут жить в~одной комнате.)
\\
\sp
Докажите, что пианисты не~смогут расселяться неограниченно далеко.
\\
\sp
Докажите, что когда-нибудь переселения прекратятся.

\item
Барон Мюнхгаузен рассказывает историю про одну очень увлекательную игру.
Он~не~помнит правил, но~точно помнит, что в~неё играют два игрока.
Кроме того, он~помнит, что в~каждой игровой ситуации у~игроков одни и~те~же
доступные варианты ходов, каждый из~которых меняет текущее состояние
на~какое-то другое, а~всего состояний конечное число.
А~проигрывает тот, кто не~может сделать ход.
Барон уверен, что эта игра точно закончится через конечное число ходов, как~бы
ни~действовали игроки.
Из~всего этого Мюнхгаузен делает вывод, что у~одного из~игроков есть выигрышная
стратегия.
Прав~ли он?

\end{problems}

