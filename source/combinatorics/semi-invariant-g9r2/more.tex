% $date: 2014-11-13
% $timetable:
%   g9r2: {}

% $groups$matter$into: false
% $groups$delegate$into: false

\section*{Мини-серия по полуинварианту}

% $authors:
% - Владимир Брагин

\begin{problems}

\item
На~доске написано несколько натуральных чисел.
Каждую минуту выбираются два числа $x$ и~$y$ и~заменяются на~$x - 2$ и~$y + 1$.
Докажите, что рано или поздно на~доске появится отрицательное число.

\item
В~долине живут фермеры.
Некоторые из~них видят дома друг друга.
Дом каждого из~них покрашен в~серый или фиолетовый цвет.
Каждое утро один из~фермеров, если видит, что цвет его дома не~совпадает с~цветом
большинства из~видимых им~домов, перекрашивает свой дом.
Докажите, что рано или поздно перекрашивания прекратятся.

\item
По~окружности выписаны $n$ натуральных чисел.
Между каждыми двумя соседними числами вписывается их~наибольший общий делитель.
После этого прежние числа стирают, а~с~оставшимися проделывают ту~же операцию.
Докажите, что через несколько шагов все числа на~окружности будут равны.

\end{problems}

