% $date: 2014-11-12
% $timetable:
%   g9r2: {}

% $groups$matter$into: false
% $groups$delegate$into: false

% $matter[-contained,no-header]:
% - verbatim: \section*{Дискретная непрерывность}
% - .[contained]

% $authors:
% - Владимир Брагин

\emph{Прыгая на~1~метр, невозможно перепрыгнуть яму шириной 2~метра\ldots}

\begin{problems}

\item
Журнал <<Юный диверсант>> выходит нерегулярно~--- два или три раза в~год.
На~обложке стоит номер журнала и~год выпуска:
№1~--- 2001, №2~--- 2001, №3~--- 2002, \ldots{}
Докажите, что если редакцию не~поймают, то~рано или поздно выйдет номер, где
два числа на~обложке совпадут.

\item
Существует~ли 1000 последовательных натуральных чисел, среди которых ровно
5~простых?

\item
В~некоторых клетках квадратной таблицы $50 \times 50$ расставлены числа $+1$
и~$-1$ таким образом, что сумма всех чисел в~таблице по~абсолютной величине
не~превосходит 100.
Докажите, что в~некотором квадрате $25 \times 25$ сумма чисел по~абсолютной
величине не~превосходит 25.

\item
На~плоскости отмечено 2014 точек.
Обязательно~ли существует круг, внутри которого ровно 100 из~них?

\item
\sp
В~комнате $2n$ юношей и~$2n$ девушек.
Всегда~ли можно провести прямую черту по~полу так, чтобы по~каждую стороную
от~черты было $n$~юношей и~$n$~девушек?
\\
\sp
Та~же задача, только никакие три человека не~стоят на~одной прямой.

\item
На~плоскости отмечено 2016 точек общего положения.
Обязательно~ли их~можно разбить на~семерки так, чтобы семиугольники, вершинами
которых служили~бы точки из~одной семерки, не~имели общих точек?

\end{problems}

