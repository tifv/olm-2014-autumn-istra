% $date: 2014-11-11
% $timetable:
%   gX:
%     2014-11-11:
%       2:

% $caption: Метод локальных изменений

\section*{Комбинаторика\footnote{Метод локальных изменений}}

% $authors:
% - Олег Орлов

\begin{problems}

\itemy{0}
На~отрезке~$AB$ отмечено $2n$ точек, симметричных относительно середины~$AB$.
При этом $n$ из~них покрашены в~красный цвет, оставшиеся $n$~--- в~синий.
Докажите, что сумма расстояний от~точки~$A$ до~красных точек равна сумме
расстояний от~точки~$B$ до~синих точек.

\item
В~каждой клетке таблицы $n \times n$ ($n \geq 4$) написано число $1$ или $-1$.
Произведение $n$ чисел, никакие два из~которых не~стоят в~одной строке или
столбце, назовём \emph{основным} произведением.
Через $S$ обозначим сумму всевозможных основных произведений данной
таблицы.
Докажите, что $S$ делится на~$4$.

\item
Рассмотрим перестановку $(a_1, a_2, \ldots, a_{20})$ чисел $1, 2, \ldots, 20$
(среди $a_1, a_2, \ldots, a_{20}$ нет одинаковых).
За~один ход разрешается поменять два числа местами.
Нашей целью является получить набор $(1, 2, \ldots, 20)$
из~$(a_1,a_2, \ldots, a_{20})$.
Для перестановки $a = (a_1, a_2, \ldots, a_{20})$ обозначим минимальное число
ходов, которые приводят к~цели из~перестановки~$a$ через $k_a$.
Найдите максимально возможное значение~$k_a$.

\item
Найдите максимальное количество ориентированных циклов длины~$3$ в~турнире
из~$14$~вершин (\emph{турниром} называется полный направленный граф).

\item
Пусть число $2008$ представлено в~виде суммы нескольких различных натуральных
слагаемых.
Чему равно максимально возможное произведение чисел, составляющих такое
разбиение?

\end{problems}

