% $date: 2014-11-11
% $timetable:
%   g10r2:
%     2014-11-11:
%       3:

\section*{Инварианты}

% $authors:
% - Юлий Тихонов
%% - Иван Митрофанов

% $matter[-preamble-package-guard]:
% - preamble package: wrapfig
% - .[preamble-package-guard]

\begin{problems}

\item
На клетке \texttt{a2} стоит белый конь, на клетке \texttt{g8} черный конь.
Ходят по очереди.
На одном из ходов один из коней был съеден другим.
Какой конь остался на доске?

\item
У Буратино $100$ круглых монет, $101$ квадратная и $102$ треугольная.
В обменном пункте, организованном котом Базилио, можно обменять две монеты
разной формы на монету третьей формы.
После серии обменов у Буратино осталась одна монета.
Какая?

\item
На доске написано число, состоящее из $100$ цифр $7$.
Каждую минуту последняя цифра числа запоминается, затем стирается и умноженная
на 5 прибавляется к тому числу, что осталось на доске после стирания.
Может ли на доске оказаться число $99$?

\item
Есть три кучки камней: в первой $51$ камень, во второй $49$, а в третьей $5$.
Разрешается объединять любые кучки в одну, а также разделять кучку, состоящую
из четного количества камней, на две равные.
Можно ли получить $105$ кучек по одному камню в каждой?

\end{problems}

\begin{wrapfigure}{L}{0.30\textwidth}\begin{center} % XXX
\vspace{-2.5ex}
\hspace{0.05\textwidth}\jeolmfigure[width=0.20\textwidth]{rooks}
\vspace{-2.5ex}
\end{center}\end{wrapfigure}

\problem
В трех левых нижних клетках доски $8 \times 8$ стоят ладьи.
За один ход разрешается передвигать ладью по обычным правилам так, чтобы после
каждого хода каждая из ладей была бы под защитой какой-нибудь другой ладьи.
Может ли после некоторого хода каждая ладья оказаться в в клетке, симметричной
исходной относительно диагонали \texttt{a8}--\texttt{h1}?

%\begin{center}
%\begin{tabular}{|c|c|c|c|c|c|c|c|}
%\hline
%a8 &  &  $ \:\:$& $ \:\:$ & $ \:\:$ & $ \:\:$ & 1 & 2 \\
%\hline
% &  &  &  &  &  &  & 3 \\
%\hline
% &  &  &  &  &  &  &  \\
%\hline
% &  &  &  &  &  &  &   \\
%\hline
% &  &  &  &  &  &  &    \\
%\hline
% &  &  &  &  &  &  &  \\
%\hline
%1 &  &  &  &  &  &  & \\
%\hline
%2 & 3 &  &  &  &  &  & h1 \\
%\hline
%\end{tabular}
%\end{center}

\begin{problems}

\item
Дана некоторая тройка чисел.
С любыми двумя из них разрешается проделывать следующее:
если эти числа равны $a$ и $b$, то их можно заменить на
$\frac{a + b}{\sqrt{2}}$ и $\frac{a - b}{\sqrt{2}}$.
Можно ли с помощью таких операций получить тройку
$(1 - \sqrt{2}, 2, \sqrt{2})$ из тройки $(1, \sqrt{2}, 1 + \sqrt{2})$?

\item
На плоскости внутри квадрата $1 \times 1$ сидят $n$ зайцев.
Если $A A' B B'$ прямоугольник, то за один ход два зайца могут прыгнуть из
точек $A$ и $B$ в точки $A'$ и $B'$.
Докажите, что никакие два зайца не удалятся друг от друга на расстояние,
большее
\quad
\sp $n$
\quad
\sp $2 \sqrt{n}$.

\end{problems}

