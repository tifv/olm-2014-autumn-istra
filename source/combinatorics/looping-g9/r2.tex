% $date: 2014-11-15
% $timetable:
%   g9r2:
%     2014-11-15:
%       2:

\section*{Путешествие по состояниям. Зацикливание}

% $authors:
% - Владимир Брагин

\begin{problems}

\item
Пусть $A$~--- конечное множество, $f \colon A \to A$~--- некоторая функция.
Докажите, что последовательность $x, f(x), f(f(x)), \ldots$ рано или поздно
становится периодической.

\item
Метеорологическая служба Тридесятого Государства следит за~погодой уже 100 лет.
Они подразделяют погоду на~дождливую или солнечную.
Метеорологи утверждают, что погода на~следующий день однозначно определяется
погодой в~предыдущие 7~дней.
Последняя неделя в~Тридесятом Государстве была полностью дождливая.
Докажите, что такое уже было и~еще будет.

\item
Кубик Рубика вывели из~исходного состояния некоторой последовательностью
поворотов граней.
Докажите, что если повторять эту последовательность поворотов достаточно долго,
то~кубик в~конце концов вернется в~исходное состояние.

\item
Докажите, что найдется число Фибоначчи, большее миллиона, дающее остаток~8
при делении на~2014.

\item
На~столе лежат 2014 спичек.
За~один ход можно взять из~кучи 1, 10 или 11 спичек.
Выигрывает тот, кто взял последнюю спичку.
Кто выиграет при правильной игре?

\item
Натуральное число заменяют суммой квадратов его цифр.
Докажите, что для любого натурального числа после некоторого количества таких
операций процесс зациклится.

\item
На~бесконечной в~обе стороны ленте записан текст на~русском языке.
Известно, что в~этом тексте число различных кусков из~15 символов равно числу
различных кусков из~16 символов.
Докажите, что на~ленте записан <<периодический>> текст, например:
\textsf{\ldots{мамамыларамумамамылараму}\ldots}

\item
\sp
Докажите, что всякая обыкновенная дробь процессом деления числителя
на~знаменатель обращается в~бесконечную периодическую десятичную дробь.
\\
\sp
Докажите, что если натуральное число~$n$ не~делится ни~на~2, ни~на~5,
то~десятичное разложение дроби $1/n$ не~имеет предпериода
(то~есть зацикливание начинается с~первого~же знака после запятой).
\\
\sp
Докажите, что длина периода десятичной дроби $1 / p$, где число $p$~---
простое, есть делитель числа $p - 1$.

\item
По~кругу сидят 2014 хамелеонов, каждый либо красного, либо зеленого цвета.
Каждую секунду каждый хамелеон перекрашивается в~том и~только в~том случае,
когда его соседи разных цветов.
Докажите, что когда-то первоначальная ситуация повторится.

\end{problems}

