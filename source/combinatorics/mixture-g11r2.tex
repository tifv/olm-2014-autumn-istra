% $date: 2014-11-18
% $timetable:
%   g11r2:
%     2014-11-18:
%       3:

\section*{Комбинаторный разнобой}

% $authors:
% - Антон Гусев

\begin{problems}

\item
Есть куча монет.
Известно, что настоящих среди них больше, чем фальшивых, все настоящие монеты
весят одинаково.
Любая фальшивая монета отличается по~весу от~настоящей, но~фальшивые монеты
могут иметь разный вес.
Мы можем пользоваться чашечными весами, владелец которых после каждого
взвешивания забирает себе (в~качестве нашей платы) любую выбранную им монету
из~взвешенных.
Докажите, что можно выделить хотя~бы одну настоящую монету, которая останется
у~нас.

\item
Дан полный граф на~$n$ вершинах.
За~ход разрешается найти какие-то $4$~ребра, образующие цикл, и~стереть одно
из~них.
Какое наименьшее количество ребер могло остаться?

\item
На~острове Логики живут $90$~рыцарей и~$10$~обманщиков.
Рыцари на~все вопросы отвечают правдиво, а~обманщики могут как сказать правду,
так и~соврать.
Разрешается выбрать любое множество жителей острова и~спросить любого
аборигена, есть~ли в~этом множестве обманщики.
Докажите, что $10$ вопросов достаточно для того, чтобы определить хотя~бы одного
рыцаря.

\item
На~плоскости даны такие $3 n$~точек, что все попарные расстояния между ними
не~больше $1$.
Докажите, что их можно разделить на~$n$ треугольников (групп по~$3$ точки) так,
чтобы сумма их площадей была не~больше $1 / 2$.

\item
В~игре <<Последний людоед>> на~необитаемый остров высаживают $16$~участников.
Каждый из~людоедов считает некоторых своих конкурентов вкусными, а~других
невкусными, и~не~меняет своего мнения.
(Сам себя людоед считает невкусным).
Каждое утро проводится голосование: каждый участник пишет список вкусных,
по~его мнению, обитателей острова, и~тех участников игры, которые признаны
вкусными не~менее, чем половиной людоедов, съедают.
Докажите, что на~девятый день никого не~съедят.

\item
На~съезде учителей математики делегаты сидели в~50 рядов по~100 человек
(расположенных в~виде прямоугольника).
Ни~у~одного из~делегатов нет в~карманах долларов и~рублей одновременно.
Каждый делегат выяснил, что у~всех его соседей справа, слева, спереди и~сзади
в~сумме столько~же рублей, сколько и~долларов.
Докажите, что у~каждого делегата в~карманах нет ни~долларов, ни~рублей.

\item
На~прямой через равные промежутки отмечены $2843546$ точек.
Петя раскрашивает половину из~них в~красный цвет, а~остальные – в~синий.
Затем Вася разбивает их на~пары <<красная>>--<<синяя>> так, чтобы сумма
расстояний между точками в~парах была максимальной.
Докажите, что этот максимум не~зависит от~того, какую раскраску сделал Петя.

\end{problems}

